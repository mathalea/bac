
\medskip

\textbf{Partie A : Détermination d'une fonction $f$ et résolution d'une équation différentielle}

\medskip

On considère la fonction $f$ définie sur $\R$ par :

\[f(x) = \e^x+ ax + b\e^{-x}\]

où $a$ et $b$ sont des nombres réels que l'on propose de déterminer dans cette partie.

Dans le plan muni d'un repère d'origine O, on a représenté ci-dessous la courbe $\mathcal{C}$, représentant la fonction $f$, et la tangente $(T)$ à la courbe $\mathcal{C}$ au point d'abscisse $0$.

\begin{center}
\psset{unit=1.25cm}
\begin{pspicture*}(-2.2,-0.6)(3.2,6.6)
\psgrid[gridlabels=0pt,subgriddiv=5,gridwidth=0.25pt,subgridwidth=0.1pt]
\psaxes[linewidth=1.25pt,labelFontSize=\scriptstyle]{->}(0,0)(-2.2,-0.4)(3.2,6.6)
\psplot[plotpoints=2000,linewidth=1.25pt,linecolor=red]{-2}{2.5}{2.71828 x exp x sub 2 2.71828 x exp div add} \uput[r](2,5){\red $\mathcal{C}$}
\psplot[linewidth=1.25pt]{-2}{2}{3 2 x mul sub}
\end{pspicture*}
\end{center}

\smallskip

\begin{enumerate}
\item Par lecture graphique, donner les valeurs de $f(0)$ et de $f’(0)$.
\item En utilisant l'expression de la fonction $f$, exprimer $f(0)$ en fonction de $b$ et en déduire la valeur de $b$.
\item On admet que la fonction $f$ est dérivable sur $\R$ et on note $f'$ sa fonction dérivée. 
	\begin{enumerate}
		\item Donner, pour tout réel $x$, l'expression de $f'(x)$.
		\item Exprimer $f'(0)$ en fonction de $a$.
		\item En utilisant les questions précédentes, déterminer $a$, puis en déduire l'expression de $f(x)$.
	\end{enumerate}
\item On considère l'équation différentielle :

\[(E):\quad  y' +y =2\e^x - x - 1\]

	\begin{enumerate}
		\item Vérifier que la fonction $g$ définie sur $\R$ par :
		
\[g(x) = \e^x - x + 2\e^{-x}.\]

est solution de l'équation $(E)$.
		\item Résoudre l'équation différentielle $y' + y = 0$.
		\item En déduire toutes les solutions de l'équation $(E)$. 
est solution de l'équation (E).
	\end{enumerate}
\end{enumerate}

\bigskip

\textbf{Partie B : Étude de la fonction $g$ sur $[1~;~+\infty[$}

\medskip

\begin{enumerate}
\item Vérifier que pour tout réel $x$, on a :

\[\e^{2x} - \e^x - 2 = \left(\e^x - 2\right)\left(\e^x + 1\right)\]

\item En déduire une expression factorisée de $g'(x)$, pour tout réel $x$.
\item On admettra que, pour tout $x \in  [1~;~ +\infty[$,\, $\e^x - 2 > 0$.

 Étudier le sens de variation de la fonction $g$ sur $[1~;~ +\infty[$.
\end{enumerate}

