
\medskip
\framebox{
\begin{minipage}[]{5.25cm}
\textbf{Principaux domaines abordés}\\
Logarithme\\
Dérivation, convexité, limites
\end{minipage}}

\medskip

Sur le graphique ci-dessous, on a représenté dans un repère orthonormé :
\begin{itemize}
\item la courbe représentative $\mathcal{C}_f$ d’une fonction $f$ définie et dérivable sur $]0~;~+\infty[$ ;
\item la tangente $\mathcal{T}_A$ à la courbe $\mathcal{C}_f$ au point A de coordonnées $\left(\dfrac{1}{\e}~;~\e\right)$ ;
\item la tangente $\mathcal{T}_B$ à la courbe $\mathcal{C}_f$ au point B de coordonnées (1~;~2).
\end{itemize}
La droite $\mathcal{T}_A$ est parallèle à l’axe des abscisses. La droite $\mathcal{T}_B$ coupe l’axe des abscisses au point de coordonnées (3~;~0) et l’axe des ordonnées au point de coordonnées (0~;~3).

\begin{center}

\psset{xunit=1.8cm,yunit=1.8cm,labelFontSize=\scriptstyle,comma=true,labelsep=0.1pt}
\begin{pspicture}(-0.4,-0.7)(8,3.7)
\multido{\n=-0.0+0.5}{16}{\psline[linewidth=0.35pt,linecolor=lightgray](\n,-0.5)(\n,3.5)}
%\multido{\n=0+1}{51}{\psline[linewidth=0.3pt](\n,-100)(\n,100)}
\multido{\n=-0.5+0.5}{8}{\psline[linewidth=0.35pt,linecolor=lightgray](0,\n)(7.50,\n) }
\psaxes[linewidth=0.95pt,Dx=0.5,Dy=0.5]{->}(0,0)(0,-0.5)(7.6,3.6)
\psaxes[linewidth=0.95pt,Dx=0.5,Dy=0.5](0,0)(0,-0.5)(7.6,3.6)
\psplot[linewidth=1.25pt,linecolor=blue,plotpoints=5000]{0.127}{7.50}{x ln 2 add x div}
\psplot[linewidth=0.85pt,linecolor=cyan,plotpoints=5000]{-0.1}{7.50}{2.71828}
\psplot[linewidth=0.85pt,linecolor=cyan,plotpoints=5000]{-0.1}{3.50}{x neg 3 add}
\psdots[dotstyle=Bullet,dotscale =1.1](0.367879,2.71828)(1,2)
\uput[ur](0.468,2.81828){A}\uput[ur](1.1,2.1){B}
\uput[u](6.5,2.5){\cyan $\mathcal{T}_A$}
\uput[r](2.2,0.5){\cyan $\mathcal{T}_B$}
\uput[r](5,0.5) {\blue $\mathcal{C}_f$}
\end{pspicture}
\end{center}

\bigskip

On note $f'$ la fonction dérivée de $f$.

\medskip

\textbf{\textsc{Partie} I}

\medskip

\begin{enumerate}
\item  Déterminer graphiquement les valeurs de $f'\left(\dfrac{1}{\e}\right)$ et de $f'(1)$.
\item En déduire une équation de la droite $\mathcal{T}_B$.
\end{enumerate}

\textbf{\textsc{Partie} II}

\medskip

On suppose maintenant que la fonction $f$ est définie sur $]0~;~+\infty[$ par : 

\[f(x) =\dfrac{2+\ln(x)}{x}.\]

\begin{enumerate}
\item Par le calcul, montrer que la courbe $\mathcal{C}_f$ passe par les points A et B et qu’elle coupe l’axe des abscisses en un point unique que l’on précisera.
\item Déterminer la limite de $f(x)$ quand $x$ tend vers 0 par valeurs supérieures, et la limite de $f(x)$ quand $x$ tend vers $+\infty$.
\item Montrer que, pour tout $x\in]0~;~\infty[$,

\[f'(x)=\dfrac{-1-\ln(x)}{x^2} .\]

\item Dresser le tableau de variations de $f$ sur $]0~;~+\infty[$.
\item On note $f''$ la fonction dérivée seconde de $f$
On admet que, pour tout $x\in]0~;~+\infty[$ 

\[f''(x)=\dfrac{1+2\ln(x)}{x^3} .\]

Déterminer le plus grand intervalle sur lequel $f$ est convexe.
\end{enumerate}

\bigskip

