 
\textbf{Commun à tous les candidats}

\smallskip

%\textbf{Les questions 1. à 5. de cet exercice peuvent être traitées de façon indépendante}
%
%\medskip
%
%On considère un cube ABCDEFGH. Le point I est le milieu du segment [EF], le point J est le milieu du segment [BC] et le point K est le milieu du segment [AE].


\begin{center}
\psset{xunit=0.85cm,yunit=0.9cm}
\begin{pspicture}(6,7)
\pspolygon(0.5,0.4)(5.5,0)(5.5,5)(0.5,5.4)%ABFE
\uput[dl](0.5,0.4){A} \uput[dr](5.5,0){B} \uput[u](5.5,5){F} \uput[ul](0.5,5.4){E}
\psline(5.5,0)(8.5,1.4)(8.5,6.4)(5.5,5)%BCGF
\uput[r](8.5,1.4){C} \uput[ur](8.5,6.4){G} 
\psline(8.5,6.4)(3.5,6.8)(0.5,5.4)%GHE 
\uput[u](3.5,6.8){H} \uput[u](3,5.2){I}\uput[dr](7,0.7){J}\uput[l](0.5,2.9){K}
\psline[linewidth=1.6pt](0.5,0.4)(3,5.2)%AI
\psline[linestyle=dashed,linewidth=1.6pt](3,5.2)(7,0.7)%IJ
\psline[linestyle=dashed,linewidth=1.6pt](0.5,2.9)(3.5,6.8)%KH
\psline[linestyle=dashed](0.5,0.4)(3.5,1.8)(3.5,6.8)%ADH
\uput[ur](3.5,1.8){D}
\psline[linestyle=dashed](3.5,1.8)(8.5,1.4)%DC
\end{pspicture}
\end{center}

En prenant le repère orthonormé $\left(\text{A}~;~\vect{\text{AB}},~ \vect{\text{AD}},~ \vect{\text{AE}}\right)$ on a :

\[\text{A}(0~;~0~;~0),\, \text{B}(0~;~0~;~0),\, \text{C}(1~;~1~;~0),\, \text{D}(0~;~1~;~0),\, \text{E}(0~;~0~;~1)\]

\[\, \text{F}(1~;~0~;~1),\, \text{G}(1~;~1~;~1),\, \text{H}(0~;~1~;~1),\, \text{I}(0,5~;~0~;~1),\, \text{J}(1~;~0,5~;~0),\, \text{K}(0~;~0~;~0,5)\]
\begin{enumerate}
\item %Les droites (AI) et (KH) sont-elles parallèles ? Justifier votre réponse,

On a $\vect{\text{AI}}(0,5~;~0~;~1)$ et $\vect{\text{KH}}(0~;~1~;~0,5)$ : ces vecteurs ne sont pas colinéaires, donc les droites (AI) et (KH) ne sont pas parallèles.
%Dans la suite, on se place dans le repère orthonormé $\left(\text{A}~;~\vect{\text{AB}},~ \vect{\text{AD}},~ \vect{\text{AE}}\right)$.

\item 
	\begin{enumerate}
		\item %Donner les coordonnées des points I et J.
Voir plus haut.
		\item %Montrer que les vecteurs $\vect{\text{IJ}},~\vect{\text{AE}}$ et $\vect{\text{AC}}$ sont coplanaires.
On a $\vect{\text{IJ}}(0,5~;~0,5~;~-1), \, \vect{\text{AE}}(0~;~0~;~1)\, \vect{\text{AC}}(1~;~1~;~0)$.

On a $2\vect{\text{IJ}} + 2\vect{\text{AE}} = \vect{\text{AC}}$.

Le vecteur $\vect{\text{AC}}$ est donc une combinaison des vecteurs $\vect{\text{IJ}}$ et $\vect{\text{AE}}$ : ces trois vecteurs sont donc coplanaires.
	\end{enumerate}
%\end{enumerate}
	
%On considère le plan $\mathcal P$ d'équation $x + 3y - 2z + 2 = 0$ ainsi que les droites $d_1$ et $d_2$ définies par les représentations paramétriques ci-dessous:
%
%\[d_1  : \left\{\begin{array}{l c l}
%x	&=&3 + t\\
%y 	&=& 8 - 2t\\
%z	&=& - 2 + 3t\\
%\end{array}\right. , t \in \R\quad \text{et}\quad 
%d_2  : \left\{\begin{array}{l c l}
%x	&=&4 + t\\
%y 	&=&1 + t\\
%z	&=&8 + 2t\\
%\end{array}\right. , t \in \R.\]
%
%\begin{enumerate}[resume]
\item %Les droites $d_1$ et $d_2$ sont-elles parallèles ? Justifier votre réponse.

$d_1$ a pour vecteur directeur $\vect{u_1}(1~;~-2~;~3)$ et $d_2$ a pour vecteur directeur $\vect{u_2}(1~;~1~;~2)$ : ces vecteurs ne sont pas colinéaires, donc les droites $d_1$ et $d_2$ ne sont pas parallèles 
\item %Montrer que la droite $d_2$ est parallèle au plan $\mathcal P$.

Le plan a pour vecteur normal le vecteur $\vect{p}(1~;~3~;~-2)$ et $d_2$ a pour vecteur directeur $\vect{u_2}(1~;~1~;~2)$.

Or $\vect{p} \cdot \vect{u_2} =  1 + 3 - 4 = 0$ : les vecteurs sont orthogonaux donc la droite $d_2$ est parallèle au plan $\mathcal P$.
\item %Montrer que le point L(4~;~0~;~3) est le projeté orthogonal du point M(5~;~3~;~1) sur le plan $\mathcal P$.
\textbf{Méthode 1}

Soit $\Delta$ la perpendiculaire à $\mathcal P$ contenant M. Cette droite a pour vecteur directeur le vecteur $\vect{p}$, donc une équation paramétrique de $\Delta$ est :

$\left\{\begin{array}{l c l}
x	&=&5 + 1t\\
y 	&=&3 + 3t\\
z	&=&1 - 2t\\
\end{array}\right., t \in \R.$

Le projeté L,  de M sur le plan $\mathcal P$ a ses coordonnées qui vérifient les quatre équations :

$\left\{\begin{array}{l c l}
x	&=&5 + 1t\\
y 	&=&3 + 3t\\
z	&=&1 - 2t\\
x + 3y - 2z + 2&=&0
\end{array}\right. , t \in \R. \Rightarrow 5 + t  + 3(3 + 3t) - 2(1 - 2t) + 2 = 0 \iff$

$5 + t + 9 + 9t - 2 + 4t + 2 = 0 \iff 14t  + 14 = 0 \iff t + 1 = 0 \iff t = - 1$.

En reportant dans les trois premières équations du système, on trouve les coordonnées de L projeté orthogonal de M sur $\mathcal P$ :

$\left\{\begin{array}{l c l}
x	&=&5 - 1\\
y 	&=&3 + 3\times (- 1)\\
z	&=&1 - 2\times (- 1)\\
\end{array}\right. \iff \left\{\begin{array}{l c l}
x	&=&4\\
y 	&=&0\\
z	&=&3\\
\end{array}\right.$

Donc le projeté orthogonal de M sur le plan $\mathcal P$ est le le point L(4~;~0~;~3).
\end{enumerate}

\textbf{Méthode 2}

On a $\vect{\text{ML}}(-1~;~- 3~;~2)$, donc $\vect{\text{ML}} = - \vect{p}$ est un vecteur normal au plan $\mathcal{P}$.

D'autre part L(4~;~0~;~3) $\in \mathcal{P} \iff 4 + 3 \times 0 - 2 \times 3 + 2 = 6 - 6 = 0$ est vraie, donc L est le projeté orthogonal de M sur le plan $\mathcal P$.
\bigskip


