
\medskip

Dans une école de statistique, après étude des dossiers des candidats, le recrutement se fait de deux façons :

\setlength\parindent{1cm}
\begin{itemize}
\item[$\bullet~~$] 10\,\% des candidats sont sélectionnés sur dossier. Ces candidats doivent ensuite passer un oral à l'issue duquel 60\,\% d'entre eux sont finalement admis à l'école.
\item[$\bullet~~$] Les candidats n'ayant pas été sélectionnés sur dossier passent une épreuve écrite à l'issue de laquelle 20\,\% d'entre eux sont admis à l'école.
\end{itemize}
\setlength\parindent{0cm}

\bigskip

\textbf{Partie 1}

\medskip

On choisit au hasard un candidat à ce concours de recrutement. On notera:

\setlength\parindent{1cm}
\begin{itemize}
\item[$\bullet~~$] $D$ l'évènement \og le candidat a été sélectionné sur dossier \fg{} ;
\item[$\bullet~~$] $A$ l'évènement \og le candidat a été admis à l'école \fg{} ;
\item[$\bullet~~$] $\overline{D}$ et $\overline{A}$ les évènements contraires des évènements $D$ et $A$ respectivement.
\end{itemize}
\setlength\parindent{0cm}

\medskip

\begin{enumerate}
\item Traduire la situation par un arbre pondéré.
\item Calculer la probabilité que le candidat soit sélectionné sur dossier et admis à l'école.
\item Montrer que la probabilité de l'évènement  $A$ est égale à $0,24$.
\item On choisit au hasard un candidat admis à l'école. Quelle est la probabilité que son dossier n'ait pas été sélectionné?
\end{enumerate}

\bigskip

\textbf{Partie 2}

\medskip

\begin{enumerate}
\item On admet que la probabilité pour un candidat d'être admis à l'école est égale à $0,24$.

On considère un échantillon de sept candidats choisis au hasard, en assimilant ce choix à un tirage au sort avec remise. On désigne par $X$ la variable aléatoire dénombrant les candidats admis à l'école parmi les sept tirés au sort.
	\begin{enumerate}
		\item On admet que la variable aléatoire $X$ suit une loi binomiale. Quels sont les paramètres de cette loi?
		\item Calculer la probabilité qu'un seul des sept candidats tirés au sort soit admis à l'école. On donnera une réponse arrondie au centième.
		\item Calculer la probabilité qu'au moins deux des sept candidats tirés au sort soient admis à cette école. On donnera une réponse arrondie au centième.
	\end{enumerate}
\item  Un lycée présente $n$ candidats au recrutement dans cette école, où $n$ est un entier naturel non nul.

On admet que la probabilité pour un candidat quelconque du lycée d'être admis à l'école est égale à $0,24$ et que les résultats des candidats sont indépendants les uns des autres.
	\begin{enumerate}
		\item Donner l'expression, en fonction de $n$, de la probabilité qu'aucun candidat issu de ce lycée ne soit admis à l'école.
		\item À partir de quelle valeur de l'entier $n$ la probabilité qu'au moins un élève de ce lycée soit admis à l'école est-elle supérieure ou égale à $0,99$ ?
	\end{enumerate}
\end{enumerate}

\bigskip

