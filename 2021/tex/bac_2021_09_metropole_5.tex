
\medskip

\begin{tabular}{|l|}\hline
Principaux domaines abordés:\\
Fonction logarithme.\\ \hline
\end{tabular}

\bigskip

\textbf{Partie I}

\medskip

On considère la fonction $h$ définie sur l'intervalle $]0~;~ +\infty[$ par:

\[h(x) = 1 + \dfrac{\ln (x)}{x}.\]

\smallskip

\begin{enumerate}
\item Déterminer la limite de la fonction $h$ en $0$.
\item Déterminer la limite de la fonction $h$ en $+\infty$.
\item On note $h'$ la fonction dérivée de $h$. Démontrer que, pour tout nombre réel $x$ de $]0~;~ +\infty[$, on a:

\[h'(x) = \dfrac{1 - \ln (x)}{x^2}.\]

\smallskip

\item Dresser le tableau de variations de la fonction $h$ sur l'intervalle $]0~;~ +\infty[$.
\item Démontrer que l'équation $h(x) = 0$ admet une unique solution $\alpha$ dans $]0~;~ +\infty[$. 

Justifier que l'on a : $0,5 < \alpha < 0,6$.
\end{enumerate}

\bigskip

\textbf{Partie II}

\medskip

Dans cette partie, on considère les fonctions $f$ et $g$ définies sur $]0~;~ +\infty[$ par : 

\[f(x) = x \ln (x) - x ;\qquad  g(x) = \ln (x).\]

On note $\mathcal{C}_f$ et $\mathcal{C}_g$ les courbes représentant respectivement les fonctions $f$ et $g$ dans un repère orthonormé \Oij.

Pout tout nombre réel $a$ strictement positif, on appelle:

\setlength\parindent{1cm}
\begin{itemize}
\item[$\bullet~~$] $T_a$ la tangente à $\mathcal{C}_f$ en son point d'abscisse $a$ ;
\item[$\bullet~~$] $D_a$ la tangente à $\mathcal{C}_g$ en son point d'abscisse $a$.
\end{itemize}
\setlength\parindent{0cm}

Les courbes  $\mathcal{C}_f$ et $\mathcal{C}_g$ ainsi que deux tangentes $T_a$ et $D_a$ sont représentées ci-dessous.

\begin{center}
\psset{unit=1cm}
\begin{pspicture*}(-0.6,-2)(7,6)
\psgrid[gridlabels=0,subgriddiv=1,gridwidth=0.06pt]
\psaxes[linewidth=1.25pt,labelFontSize=\scriptstyle](0,0)(0,-2)(6.9,5.9)
\psaxes[linewidth=1.25pt,labelFontSize=\scriptstyle]{->}(0,0)(1,1)
\psplot[plotpoints=2000,linewidth=1.25pt,linecolor=red]{0.01}{7}{x ln x mul x sub}
\psplot[plotpoints=2000,linewidth=1.25pt,linecolor=blue]{0.01}{7}{x ln}
\psplot[plotpoints=2000,linewidth=1.25pt,linecolor=red,linestyle=dashed]{0.01}{7}{1.833 x mul 6.254 sub}
\psplot[plotpoints=2000,linewidth=1.25pt,linecolor=blue,linestyle=dashed]{0.01}{7}{0.162 x mul 0.833 add}
\uput[u](1,1){\blue $D_a$}\uput[r](2.9,-1){\red $T_a$}\uput[l](4.3,2){\red $\mathcal{C}_f$ }
\uput[r](0.2,-1.8){\blue $\mathcal{C}_g$}
%\psplotTangent[linecolor=red]{6.25}{4cm}{x ln}
%\psplotTangent[linecolor=blue]{6.25}{4cm}{1 x div}
\psline[linestyle=dotted,linewidth=1.25pt](6.25,0)(6.25,5.204)
\uput[d](6.25,0){$a$}
\end{pspicture*}
\end{center}

On recherche d'éventuelles valeurs de $a$ pour lesquelles les droites $T_a$ et $D_a$ sont perpendiculaires. 

Soit $a$ un nombre réel appartenant à l'intervalle $]0~;~ +\infty[$.


\medskip

\begin{enumerate}
\item Justifier que la droite $D_a$ a pour coefficient directeur $\dfrac{1}{a}$.
\item Justifier que la droite $T_a$ a pour coefficient directeur $\ln (a)$.
\end{enumerate}

On rappelle que dans un repère orthonormé, deux droites de coefficients directeurs respectifs $m$ et $m'$sont perpendiculaires si et seulement si $mm' = -1$.

\begin{enumerate}[resume]
\item Démontrer qu'il existe une unique valeur de $a$, que l'on identifiera, pour laquelle les droites $T_a$ et $D_a$ sont perpendiculaires.
\end{enumerate}

%%%%%%%%%%%%%%%%%%% sujet J2 %%%%%%%%%%%%%%%%%%%%%%%%%%%%
