\textbf{\large Exercice A}

\medskip

\begin{tabularx}{\linewidth}{|X|}\hline
\textbf{Principaux domaines abordés: Fonction exponentielle ; dérivation.}\\ \hline
\end{tabularx}

\bigskip

\parbox{0.35\linewidth}{Le graphique ci-contre représente, dans un repère orthogonal, les courbes 
$\mathcal{C}_f$ et $\mathcal{C}_g$ des fonctions $f$ et $g$ définies sur $\R$ par :

\[f(x) = x^2\text{e}^{-x}\quad \text{et} \quad g(x) = \text{e}^{-x}.\]}
\hfill \parbox{0.55\linewidth}{\psset{xunit=1.6cm,yunit=0.8cm}
\begin{pspicture*}(-2.5,-1)(2.5,9)
\multido{\n=0+2}{5}{\psline[linewidth=0.1pt](-2.5,\n)(2.5,\n)}
\multido{\n=-2+1}{5}{\psline[linewidth=0.1pt](\n,-1)(\n,9)}
\psaxes[linewidth=1.25pt,labelFontSize=\scriptstyle,Dy=2]{->}(0,0)(-2.5,-1)(2.5,9)
\psplot[linewidth=1.25pt,linecolor=blue,plotpoints=2000]{-2.5}{2.5}{x dup mul 2.71828 x exp div}
\psplot[linewidth=1.25pt,linecolor=red,plotpoints=2000]{-2.5}{2.5}{2.71828 x neg exp}
\psline[linestyle=dashed,linewidth=1.25pt](-0.6,-1)(-0.6,9)
\psdots(-0.6,0.656)(-0.6,1.822)
\uput[r](-0.6,0.656){\footnotesize N}\uput[ur](-0.6,1.822){\footnotesize M}
\uput[r](-1.4,7){\blue $\mathcal{C}_f$}\uput[l](-2,7.2){\red $\mathcal{C}_g$}
\psplot{-2.5}{2.5}{x 2 add} \uput[ul](2,4){$\Delta$}
\end{pspicture*}}

%\textbf{La question 3 est indépendante des questions 1 et 2.}

%\medskip

\begin{enumerate}
\item 
	\begin{enumerate}
		\item Les abscisses des points d'intersection de $\mathcal{C}_f$ et $\mathcal{C}_g$ sont les solutions de l'équation $f(x)=g(x)$. On résout cette équation:
		
$f(x)=g(x) \iff x^2\e^{-x}=\e^{-x}\iff (x^2-1)\e^{-x}=0$

Pour tout réel $x$, $\e^{-x}>0$ donc $\e^{-x}\neq 0$.

$f(x)=g(x) \iff x^2-1=0 \iff x=-1 \text{ ou } x=1$.

Pour $x=-1$, $g(x)=\e$, et pour $x=1$, $g(x)=\e^{-1}$.

Les coordonnées des points d'intersection de $\mathcal{C}_f$ et $\mathcal{C}_g$ sont donc $(-1~;~\e)$ et $(1~;~\e^{-1})$.
		
		\item Pour étudier la position relative des courbes $\mathcal{C}_f$ et $\mathcal{C}_g$, on étudie le signe de $f(x)-g(x)$, c'est-à -dire de $(x^2-1)\e^{-x}$.
		
\begin{center}
{
\renewcommand{\arraystretch}{1.5}
\def\esp{\hspace*{1cm}}
$\begin{array}{|c | *7{c} |} 
\hline
x  & -\infty & \esp & -1 & \esp & 1 & \esp & +\infty \\
\hline
x^2-1 &  & \pmb{+} &  \vline\hspace{-2.7pt}{0} & \pmb{-} & \vline\hspace{-2.7pt}{0} & \pmb{+} &\\
\hline
\e^{-x} &  & \pmb{+} &  \vline\hspace{-2.7pt}{\phantom 0} & \pmb{+} & \vline\hspace{-2.7pt}{\phantom 0} & \pmb{+} &\\
\hline
(x^2-1)\e^{-x} &  & \pmb{+} &  \vline\hspace{-2.7pt}{0} & \pmb{-} & \vline\hspace{-2.7pt}{0} & \pmb{+} &\\
\hline
\end{array}$
}
\end{center}		

Donc sur les intervalles $]-\infty~;~-1[$ et $]1~;~+\infty[$	, la courbe $\mathcal{C}_f$ est au dessus de la courbe $\mathcal{C}_g$,  et sur l'intervalle $]-1~;~1[$,	 la courbe $\mathcal{C}_f$ est en dessous de la courbe $\mathcal{C}_g$, 
		
	\end{enumerate}
\item  Pour tout nombre réel $x$ de l'intervalle $[-1~;~1]$, on considère les points M de coordonnées $(x~;~f(x))$ et N de coordonnées $(x~;~g(x))$, et on note $d(x)$ la distance MN. \\
On admet que : $d(x)= \e^{-x} - x^2\e^{-x}$.

On admet que la fonction $d$ est dérivable sur  $[-1~;~1]$ et on note $d'$ sa fonction dérivée.
	\begin{enumerate}
		\item %Montrer que $d'(x) = \e^{-x}\left(x^2 - 2x - 1\right)$. 
$d'(x) = (-1) \e^{-x} - \left ( 2x \times \e^{-x} + x^2 \times (-1)\e^{-x}\right )
= \e^{-x} \left ( -1 -2x +x^2\right ) = \e^{-x} \left (x^2-2x-1\right )$
				
		\item% En déduire les variations de la fonction $d$ sur l'intervalle $[-1~;~1]$.
$x^2-2x-1 = x^2-2x+1-2 = (x-1)^2-2 = \left ( x-1-\sqrt{2}\right )\left (x-1+\sqrt{2}\right )$

\begin{center}
{
\renewcommand{\arraystretch}{1.5}
\def\esp{\hspace*{0cm}}
$\begin{array}{|c | *{11}{c} |} 
\hline
x  & -\infty & \esp & -1 & \esp & 1-\sqrt{2} & \esp & 1 & \esp & 1+\sqrt{2} & \esp & +\infty \\
\hline
x-1-\sqrt{2} &  & \pmb{-} & \vline\hspace{-2.7pt}{\phantom 0} & \pmb{-} &  \vline\hspace{-2.7pt}{\phantom 0} & \pmb{-} & \vline\hspace{-2.7pt}{\phantom 0} & \pmb{-} &  \vline\hspace{-2.7pt}{0} & \pmb{+} &\\
\hline
x-1+\sqrt{2} &  & \pmb{-} & \vline\hspace{-2.7pt}{\phantom 0} & \pmb{-} &  \vline\hspace{-2.7pt}{0} & \pmb{+} & \vline\hspace{-2.7pt}{\phantom 0} & \pmb{+} &  \vline\hspace{-2.7pt}{\phantom 0} & \pmb{+} &\\
\hline
\e^{-x} &  & \pmb{+} & \vline\hspace{-2.7pt}{\phantom 0} & \pmb{+} &  \vline\hspace{-2.7pt}{\phantom 0} & \pmb{+} & \vline\hspace{-2.7pt}{\phantom 0} & \pmb{+} &  \vline\hspace{-2.7pt}{\phantom 0} & \pmb{+} &\\
\hline
\e^{-x} \left (x^2-2x-1\right )&  & \pmb{+} & \vline\hspace{-2.7pt}{\phantom 0} & \pmb{+} &  \vline\hspace{-2.7pt}{0} & \pmb{-} & \vline\hspace{-2.7pt}{\phantom 0} & \pmb{-} &  \vline\hspace{-2.7pt}{0} & \pmb{+} &\\
\hline
\end{array}$
}
\end{center}

\begin{list}{\textbullet}{}
\item Sur l'intervalle $\left [-1~;~1 - \sqrt{2}\right [$, $d'(x)>0$ donc $d$ est strictement croissante.
\item Sur l'intervalle $\left ]1-\sqrt{2}~;~1\right ]$, $d'(x)<0$ donc $d$ est strictement décroissante.
\end{list}		
		
		\item %Déterminer l'abscisse commune $x_0$ des points M$_0$ et N$_0$ permettant d'obtenir une distance $d\left(x_0\right)$ maximale, et donner une valeur approchée à  $0,1$ près de la distance M$_0$N$_0$.
D'après la question précédente, la distance $d(x)$ est maximale pour $x_0= 1-\sqrt{2}$.

Elle vaut $d\left(1-\sqrt{2}\right) = \left(1 - \left(1 + 2 - 2\sqrt{2}\right)\right)\e^{-1+\sqrt{2}} = \left(2\sqrt{2} - 2\right)\e^{-1+\sqrt{2}} \approx 1,254$, soit 1,3 au dixième près..
	\end{enumerate}
\item Soit $\Delta$ la droite d'équation $y = x + 2$.

On considère la fonction $h$ dérivable sur $\R$ et définie par: $h(x) = \e^{-x} - x - 2$.

Pour déterminer le nombre de solutions de l'équation $h(x) = 0$, on étudie la fonction $h$.

$h'(x)=-\e^{-x}-1$ donc $h'(x)<0$; la fonction $h$ est donc strictement décroissante sur $\R$.

\begin{list}{\textbullet}{}
\item $h(-1)=\e^{1}+1-2= \e-1>0$; comme $h$ est strictement décroissante, $h(x)>0$ pour $x<-1$, donc $h$ ne s'annule pas sur l'intervalle $]-\infty~;~-1[$.

\item $h(0)=\e^{0}-2 = -1 <0$; comme $h$ est strictement décroissante, $h(x)<0$ pour $x>0$, donc $h$ ne s'annule pas sur l'intervalle $]0~;~+\infty[$.

\item Sur l'intervalle $[-1~;~0]$, la fonction $h$ est continue et strictement décroissante, et on sait que $h(-1)>0$ et $h(0)<0$; donc d'après le corollaire du théorème des valeurs intermédiaires, l'équation $h(x)=0$ admet une solution unique.
\end{list}

La droite $\Delta$ et la courbe $\mathcal{C}_g$ ont donc  un unique point d'intersection dont l'abscisse est comprise entre $-1$ et $0$.

\end{enumerate}

\bigskip

