\textbf{\large Exercice B}

\medskip

\begin{tabularx}{\linewidth}{|X|}\hline
\textbf{Principaux domaines abordés : Fonction logarithme; convexité}
\\ \hline
\end{tabularx}

\medskip

On considère la fonction $f$ définie sur l'intervalle $]0~;~ +\infty[$ par :
$f(x) = x + 4 - 4\ln (x) - \dfrac{3}{x}$,

où ln désigne la fonction logarithme népérien.

On note $\mathcal{C}$ la représentation graphique de $f$ dans un repère orthonormé.

\medskip

\begin{enumerate}
\item On détermine la limite de la fonction $f$ en $+\infty$.

$f(x) = x\left ( 1- 4\dfrac{\ln(x)}{x}\right ) + 4 -  \dfrac{3}{x}$

$\left.
\begin{array}{@{} r}
\ds\lim_{x\to +\infty} \dfrac{\ln(x)}{x} = 0 \implies \ds\lim_{x\to +\infty}  x\left ( 1- 4\dfrac{\ln(x)}{x}\right ) + 4 = +\infty\\
\ds\lim_{x\to +\infty} \dfrac{3}{x}= 0
\end{array}
\right \rbrace
\implies
\ds\lim_{x\to +\infty} f(x) = +\infty$

\item On admet que la fonction $f$ est dérivable sur $]0~;~ +\infty[$ et on note $f'$ sa fonction dérivée.

%Démontrer que, pour tout nombre réel $x > 0$, on a : $f'(x) = \dfrac{x^2 - 4x + 3x^2}{x^2}.$
$f'(x)= 1 + 0 - \dfrac{4}{x} +\dfrac{3}{x^2} = \dfrac{x^2 - 4x +3}{x^2}$

\item
	\begin{enumerate}
		\item %Donner le tableau de variations de la fonction $f$ sur l'intervalle $]0~;~ +\infty[$. 
		
%On y fera figurer les valeurs exactes des extremums et les limites de $f$ en $0$ et en $+ \infty$. 
%		
%On admettra que $\displaystyle\lim_{x \to 0} f(x) = - \infty$.
On cherche le signe de $f'(x)$ sur $]0~;~+\infty[$:

$x^2-4x+3 = (x-1)(x-3)$

\begin{center}
{
\renewcommand{\arraystretch}{1.5}
\def\esp{\hspace*{2cm}}
$\begin{array}{|c | l *{6}{c} |} 
\hline
x  & 0 & \esp & 1 & \esp & 3 & \esp & +\infty \\
\hline
x^2-4x+3 &  & \pmb{+} &  \vline\hspace{-2.7pt}{0} & \pmb{-} & \vline\hspace{-2.7pt}{0} & \pmb{-} &\\
\hline
x^2 & 0  & \pmb{+} &  \vline\hspace{-2.7pt}{\phantom 0} & \pmb{+} & \vline\hspace{-2.7pt}{\phantom 0} & \pmb{+} &\\
\hline
f'(x) & \vline\;\vline  & \pmb{+} &  \vline\hspace{-2.7pt}{0} & \pmb{-} & \vline\hspace{-2.7pt}{0} & \pmb{+} & \\
\hline
\end{array}$
}
\end{center}

$f(1)=1+4-4\ln(1)-\dfrac{3}{1}=2$;
$f(3) = 3+4-4\ln(3) - \dfrac{3}{3}=6-4\ln(3)\approx 1,69$

On établit le tableau des variations de $f$ en admettant que que $\displaystyle\lim_{x \to 0} f(x) = - \infty$:

\begin{center}
{\renewcommand{\arraystretch}{1.3}
\psset{nodesep=3pt,arrowsize=2pt 3}  % paramètres
\def\esp{\hspace*{1.5cm}}% pour modifier la largeur du tableau
\def\hauteur{20pt}% mettre au moins 20pt pour augmenter la hauteur
$\begin{array}{|c|l *5{c} c|}
\hline
 x & 0 & \esp & 1 & \esp & 3 & \esp & +\infty \\
 \hline
f'(x) & \vline\;\vline\; &  \pmb{+} & \vline\hspace{-2.7pt}0 & \pmb{-} & \vline\hspace{-2.7pt}0 & \pmb{+} & \\  
\hline
  & \vline\;\vline\; &  & \Rnode{max1}{2} & & & &  \Rnode{max2}{+\infty} \\
f (x) &\vline\;\vline\; &  & & & & & \rule{0pt}{\hauteur}\\
 &\vline\;\vline\; \Rnode{min1}{-\infty} & & & & \Rnode{min2}{6-4\ln(3)\approx 1,61} & & \rule{0pt}{\hauteur} 
\ncline{->}{min1}{max1} 
\ncline{->}{max1}{min2}
\ncline{->}{min2}{max2} \\
\hline
\end{array}$
}
\end{center}

		\item %Par simple lecture du tableau de variations, préciser le nombre de solutions de l'équation $f(x) = \dfrac{5}{3}$.
$\bullet~~$$\dfrac{5}{3}\in ]-\infty~;~2]$ donc l'équation $f(x) =\dfrac{5}{3}$ admet une unique solution dans l'intervalle $]0~;~1]$.

$\bullet~~$$\dfrac{5}{3} \approx 1,67$ et $f(3) = 6 - 4\ln 3 \approx 1,61$ donc $\dfrac{5}{3} \in [6 - 4\ln 3~;~2]$, donc l'équation $f(x)=\dfrac{5}{3}$ admet une solution unique dans l'intervalle $]1~;~3[$.

$\bullet~~$$\dfrac{5}{3} \in [6 - 4\ln 3~;~+ \infty[$, donc $f(x) =\dfrac{5}{3}$ admet une unique solution dans l'intervalle $]0~;~1]$.

Conclusion : l'équation $f(x)=\dfrac{5}{3}$ admet donc trois solutions dans $]0~;~+\infty[$.

Voir cidessus les valeurs approchées des solutions.
	\end{enumerate}

\begin{center}
\psset{unit=1cm,arrowsize=2pt 3}
\begin{pspicture*}(-1,-4)(11,5)
\psgrid[gridlabels=0pt,subgriddiv=1,gridwidth=0.1pt]
\psaxes[linewidth=1.25pt,labelFontSize=\scriptstyle]{->}(0,0)(-1,-3.95)(11,4.95)
\psaxes[linewidth=1.25pt,labelFontSize=\scriptstyle](0,0)(-1,-3.95)(11,4.95)
\psplot[plotpoints=2000,linewidth=1.25pt,linecolor=red]{0.1}{10}{x 4 add x ln  4 mul sub 3 x div sub}
\psline[linecolor=blue,linewidth=0.25pt](0,1.6667)(11,1.6667)
\psline[linewidth=0.5pt,linestyle=dashed,ArrowInside=->]{->}(0,1.667)(0.61,1.667)(0.61,0)
\psline[linewidth=0.5pt,linestyle=dashed,ArrowInside=->]{->}(0.61,1.667)(2.28,1.667)(2.28,0)
\psline[linewidth=0.5pt,linestyle=dashed,ArrowInside=->]{->}(2.28,1.667)(3.78,1.667)(3.78,0)
\uput[d](0.61,0){\footnotesize $\approx 0,61$}\uput[d](2.28,0){\footnotesize $\approx 2,28$}\uput[d](3.78,0){\footnotesize $\approx 3,78$}\uput[u](9.5,1.667){\blue $y = \frac{5}{3}$}
\uput[u](9.5,4.25){\red $\mathcal{C}_f$}
\end{pspicture*}
\end{center}
\item Pour étudier la convexité de  $f$, on détermine le signe de $f''$, la dérivée seconde de $f$.

%On justifiera que la courbe $\mathcal{C}$ admet un unique point d'inflexion, dont on précisera les coordonnées.

$f'(x)=\dfrac{x^2-4x+3}{x^2}$ donc

$f''(x) = \dfrac{(2x-4)\times x^2 - (x^2-4x+3)\times 2x}{x^4}
= \dfrac{(2x^2 -4x -2x^2 +8x -6)\times x}{x^4}
=\dfrac{4x-6}{x^3}$

\begin{center}
\renewcommand{\arraystretch}{1.5}
\def\esp{\hspace*{2cm}}
$\begin{array}{|c | l *{4}{c} |} 
\hline
x  & 0 & \esp & \frac{3}{2} & \esp  & +\infty \\
\hline
4x-6 &  & \pmb{-} &  \vline\hspace{-2.7pt}{0} & \pmb{+} &    \\
\hline
x^3 &0  & \pmb{+} &  \vline\hspace{-2.7pt}{\phantom 0} & \pmb{+} &    \\
\hline
f''(x) &\vline\;\vline  & \pmb{-} &  \vline\hspace{-2.7pt}{0} & \pmb{+} &    \\
\hline
 &\vline\;\vline  & f \text{ concave} &  \vline\hspace{-2.7pt}{\phantom0} & f \text{ convexe} &    \\
\hline
\end{array}$
\renewcommand{\arraystretch}{1}
\end{center}

La dérivée seconde s'annule et change de signe pour $x=\dfrac{3}{2}$ donc la courbe $\mathcal{C}_f$ admet un unique point d'inflexion d'abscisse $\dfrac{3}{2}$.

$f\left (\dfrac{3}{2}\right) = \dfrac{3}{2} +4 -4\ln\left (\dfrac{3}{2}\right ) -\dfrac{3}{\frac{3}{2}}
= \dfrac{11}{2} -4\ln\left (\dfrac{3}{2}\right ) -2
= \dfrac{7}{2} -4\ln\left (\dfrac{3}{2}\right ) $

La courbe $\mathcal{C}$ admet un unique point d'inflexion de coordonnées
$\left ( \dfrac{3}{2}~;~\dfrac{7}{2} -4\ln\left (\dfrac{3}{2}\right )\right )$.
 
\end{enumerate}
