
\textbf{Commun à tous les candidats}

\medskip

On considère la suite $\left(u_n\right)$ définie par $u_0 = \np{10000}$ et pour tout entier naturel $n$ : 

\[u_{n+1} = 0,95u_n + 200.\]

\begin{enumerate}
\item Calculer $u_1$ et vérifier que $u_2 = \np{9415}$. 
\item 
	\begin{enumerate}
		\item Démontrer, à l'aide d'un raisonnement par récurrence, que pour tout entier naturel $n$ :

\[u_n > \np{4000}.\]

		\item On admet que la suite $\left(u_n\right)$ est décroissante. Justifier qu'elle converge.
	\end{enumerate}
\item  Pour tout entier naturel $n$, on considère la suite $\left(v_n\right)$ définie par: $v_n = u_n - \np{4000}$.
	\begin{enumerate}
		\item Calculer $v_0$.
		\item Démontrer que la suite $\left(v_n\right)$ est géométrique de raison égale à $0,95$.
		\item En déduire que pour tout entier naturel $n$ :

\[u_n = \np{4000} + \np{6000} \times 0,95^n.\]

		\item Quelle est la limite de la suite $\left(u_n\right)$ ? Justifier la réponse.
	\end{enumerate}
\item En 2020, une espèce animale comptait \np{10000} individus. L'évolution observée les années précédentes conduit à estimer qu'à partir de l'année 2021, cette population baissera de 5\,\% chaque début d'année.

Pour ralentir cette baisse, il a été décidé de réintroduire $200$ individus à la fin de chaque année, à partir de 2021.

Une responsable d'une association soutenant cette stratégie affirme que : \og l'espèce ne devrait pas s'éteindre, mais malheureusement, nous n'empêcherons pas une disparition de plus de la moitié de la population \fg.

Que pensez-vous de cette affirmation ? Justifier la réponse.
\end{enumerate}

\bigskip

