
\textbf{Commun à tous les candidats}

\medskip

Un test est mis au point pour détecter une maladie dans un pays.

Selon les autorités sanitaires de ce pays, 7\,\% des habitants sont infectés par cette maladie. 

Parmi les individus infectés, 20\,\% sont déclarés négatifs.

Parmi les individus sains, 1\,\% sont déclarés positifs.

Une personne est choisie au hasard dans la population.

On note :

\setlength\parindent{9mm}
\begin{itemize}
\item[$\bullet~~$]$M$ l'évènement : \og la personne est infectée par la maladie\fg{} ;
\item[$\bullet~~$]$T$ l'évènement: \og le test est positif \fg.
\end{itemize}
\setlength\parindent{0mm}

\medskip

\begin{enumerate}
\item Construire un arbre pondéré modélisant la situation proposée.
\item 
	\begin{enumerate}
		\item Quelle est la probabilité pour que la personne soit infectée par la maladie et que son test soit positif?
		\item Montrer que la probabilité que son test soit positif est de \np{0,0653}.
	\end{enumerate}
\item On sait que le test de la personne choisie est positif. 

Quelle est la probabilité qu'elle soit infectée ?

On donnera le résultat sous forme approchée à $10^{-2}$ près.
\item On choisit dix personnes au hasard dans la population. La taille de la population de ce pays permet d'assimiler ce prélèvement à un tirage avec remise.

On note $X$ la variable aléatoire qui comptabilise le nombre d'individus ayant un test positif parmi les dix personnes.
	\begin{enumerate}
		\item Quelle est la loi de probabilité suivie par $X$ ? Préciser ses paramètres.
		\item Déterminer la probabilité pour qu'exactement deux personnes aient un test positif.

On donnera le résultat sous forme approchée à $10^{-2}$ près.
	\end{enumerate}
\item Déterminer le nombre minimum de personnes à tester dans ce pays pour que la probabilité qu'au moins une de ces personnes ait un test positif, soit supérieure à 99\,\%.
\end{enumerate}

\bigskip

