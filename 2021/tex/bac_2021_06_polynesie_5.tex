\textbf{EXERCICE B}

\medskip

\fbox{\textbf{Principaux domaines abordés: Fonction logarithme népérien, dérivation}}

\medskip

Cet exercice est composé de deux parties.

\medskip

Certains résultats de la première partie seront utilisés dans la deuxième.

\medskip

\begin{center}\textbf{Partie 1 : Étude d'une fonction auxiliaire}\end{center}

Soit la fonction $f$ définie sur l'intervalle [1~;~4] par : 

\[f(x) = - 30x + 50 + 35\ln x.\]

\begin{enumerate}
\item On rappelle que $f'$ désigne la fonction dérivée de la fonction $f$.
	\begin{enumerate}
		\item Pour tout nombre réel $x$ de l'intervalle [1~;~4], montrer que: 
		
\[f'(x) = \dfrac{35- 30x}{x}.\]

		\item Dresser le tableau de signe de $f'(x)$ sur l'intervalle [1~;~4].
		\item En déduire les variations de $f$ sur ce même intervalle.
	\end{enumerate}
\item Justifier que l'équation $f(x) = 0$ admet une unique solution, notée $\alpha$, sur l'intervalle [1~;~4] puis donner une valeur approchée de $\alpha$ à $10^{-3}$ près.
\item  Dresser le tableau de signe de $f(x)$ pour $x \in [1~;~4]$.
\end{enumerate}

\bigskip

\textbf{Partie 2 : Optimisation}

\medskip

Une entreprise vend du jus de fruits. Pour $x$ milliers de litres vendus, avec $x$ nombre réel de l'intervalle [1~;~4], l'analyse des ventes conduit à modéliser le bénéfice $B(x)$ par l'expression donnée en milliers d'euros par :

\[B(x) = - 15x^2 + 15x +35x \ln x.\]

\begin{enumerate}
\item D'après le modèle, calculer le bénéfice réalisé par l'entreprise lorsqu'elle vend \np{2500}~litres de jus de fruits.

On donnera une valeur approchée à l'euro près de ce bénéfice.
\item Pour tout $x$ de l'intervalle [1~;~4], montrer que $B'(x) = f(x)$ où $B'$ désigne la fonction dérivée de $B$.
\item 
	\begin{enumerate}
		\item À l'aide des résultats de la \textbf{partie 1}, donner les variations de la fonction $B$ sur l'intervalle [1~;~4].
		\item En déduire la quantité de jus de fruits, au litre près, que l'entreprise doit vendre afin de réaliser un bénéfice maximal.
	\end{enumerate}
\end{enumerate}
