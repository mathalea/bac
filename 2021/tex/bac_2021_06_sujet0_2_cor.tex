\textbf{\large\textsc{Exercice 2} \hfill commun à tous les candidats \hfill 5 points}  

\medskip

On considère le cube ABCDEFGH de côté 1, le milieu I de [EF] et J le symétrique de E par rapport à F{}.

\begin{center}
\psset{xunit=5cm,yunit=5cm}
\begin{pspicture}(-0.1,-0.1)(2.2,1.5)
%\psgrid
\psframe(0,0)(1,1)
\psline(1,0)(1.35,0.35)
\psline[linestyle=dashed](0,0)(0.35,0.35)
\psline[linestyle=dashed](0.35,0.35)(1.35,0.35)
\psline[linestyle=dashed](0.35,0.35)(0.35,1.35)
\psline(1.35,0.35)(1.35,1.35)
\psline(0,1)(0.35,1.35)\psline(0.35,1.35)(1.35,1.35)\psline(1,1)(1.35,1.35)
\uput[dl](0,0){A} \uput[d](1,0){B}
\uput[r](1.35,0.35){C} \uput[l](0.35,0.35){D}
\uput[ul](0.35,1.35){H}
\uput[ul](0,1){E}
\uput[ur](1.35,1.35){G}
\uput[u](0.5,1){I}
\psdots(0.5,1)(2,1)
\uput[u](1,1){F}
\psline[linewidth=0.5pt,linecolor=gray] (1,1)(2,1)
\uput[r](2,1){J}
\end{pspicture}

\end{center}

\medskip

Dans tout l'exercice, l'espace est rapporté au repère orthonormé $\left(\text{A}~;~\vv{\text{AB}},~\vv{\text{AD}},~\vv{\text{AE}}\right)$.

Les sommets du cube ont pour coordonnées:
A\,$\begin{pmatrix} 0 \\ 0 \\ 0 \end{pmatrix}$,
B\,$\begin{pmatrix} 1 \\ 0 \\ 0 \end{pmatrix}$,
D\,$\begin{pmatrix} 0 \\ 1 \\ 0 \end{pmatrix}$,
E\,$\begin{pmatrix} 0 \\ 0 \\ 1 \end{pmatrix}$,
C\,$\begin{pmatrix} 1 \\ 1 \\ 0 \end{pmatrix}$,
F\,$\begin{pmatrix} 1 \\ 0 \\ 1 \end{pmatrix}$,
H\,$\begin{pmatrix} 0 \\ 1 \\ 1 \end{pmatrix}$ et
G\,$\begin{pmatrix} 1 \\ 1 \\ 1 \end{pmatrix}$.

\medskip

\begin{enumerate}
\item 
	\begin{enumerate}
		\item %Par lecture graphique, donner les coordonnées des points I et J.
\begin{list}{\textbullet}{}
\item Le point I est le milieu de [EF] donc I a pour coordonnées
$\begin{pmatrix} \frac{1}{2} \\ 0 \\ 1 \end{pmatrix}$.
\item Le point J est le symétrique de E par rapport à F{}, donc J a pour coordonnées
$\begin{pmatrix} 2 \\ 0 \\ 1 \end{pmatrix}$.
\end{list}		
				
		\item On en déduit les coordonnées des vecteurs 
		$\vvt{DJ}\,\begin{pmatrix} 2 \\ -1 \\ 1 \end{pmatrix}$,
		$\vvt{BI}\,\begin{pmatrix} -\frac{1}{2} \\ 0 \\ 1 \end{pmatrix}$ et 
		$\vvt{BG}\,\begin{pmatrix} 0 \\ 1 \\ 1 \end{pmatrix}$.
		
		\item %Montrer que $\vv{\text{DJ}}$ est un vecteur normal au plan (BGI).
\begin{list}{\textbullet}{}
\item Les vecteurs $\vvt{BI}$ et $\vvt{BG}$ ne sont pas colinéaires donc ce sont deux vecteurs directeurs du plan (BGI).
\item $\vvt{DJ}\cdot\vvt{BI} = -1+0+1=0$ donc $\vvt{DJ}\perp\vvt{BI}$.
\item $\vvt{DJ}\cdot\vvt{BG} = 0-1+1=0$ donc $\vvt{DJ}\perp\vvt{BG}$.
\end{list}

Donc le vecteur $\vvt{DJ}$ est orthogonal à deux vecteurs non colinéaires du plan (BGI), donc il est normal au plan (BGI).
		
		\item %Montrer qu'une équation cartésienne du plan (BGI) est $2x - y + z - 2 = 0$.
\begin{list}{\textbullet}{}
\item Le vecteur $\vvt{DJ}\,\begin{pmatrix} 2 \\ -1 \\ 1 \end{pmatrix}$ est normal au plan (BGI) donc le plan (BGI) a une équation de la forme $2x-y+z+d=0$.
\item Le point B appartient au plan (BGI) donc les coordonnées de B vérifient l'équation du plan; donc $2x_{\text B}-y_{\text B}+z_{\text B}+d=0$, ce qui équivaut à $2-0+0+d=0$, ce qui veut dire que $d=-2$.
\end{list}		

Donc une équation cartésienne du plan (BGI) est $2x - y + z - 2 = 0$.		
		
		
	\end{enumerate}
\item On note $d$ la droite passant par F et orthogonale au plan (BGI).
\begin{enumerate}
\item %Déterminer une représentation paramétrique de la droite $d$.
La droite $d$ est orthogonale au plan (BGI), et $\vvt{DJ}$ est un vecteur normal au plan (BGI), donc $\vvt{DJ}$ est un vecteur directeur de la droite $d$.

Le point F appartient à la droite $d$ donc la droite $d$ est l'ensemble des points M de coordonnées $(x~;~y~;~z)$ tels que $\vvt{FM}$ et $\vvt{DJ}$ soient colinéaires.

$\vvt{FM}$ et $\vvt{DJ}$ colinéaires
$\iff \vvt{FM}=t.\vvt{DJ}
\iff
\left\{
\begin{array}{r !{=} l}
x-1 & t \times 2\\
y-0 & t \times (-1)\\
z-1 & t\times 1
\end{array}
\right .$

Donc la droite $d$ a pour équation
$\left\{
\begin{array}{r !{=} r}
x&1+2t\\
y& -t\\
z &1+ \hphantom{2}t
\end{array}
\right . \,,\, t \in \R$

\item On considère le point L de coordonnées $\left(\frac{2}{3}~;~\frac{1}{6}~;~\frac{5}{6}\right)$.

%Montrer que L est le point d'intersection de la droite $d$ et du plan (BGI).

\begin{list}{\textbullet}{}
\item Pour prouver que $\text L\in d$, on cherche $t$ pour que
$\left\{
\begin{array}{r !{=} r}
\frac{2}{3}&1+2t\rule[-7pt]{0pt}{0pt}\\
\frac{1}{6}& -t\rule[-7pt]{0pt}{0pt}\\
\frac{5}{6} &1+ \hphantom{2}t
\end{array}
\right . $

On trouve $t=-\dfrac{1}{6}$ donc $\text L \in d$.

\item Le plan (BGI) a pour équation $2x-y+z-2=0$; 
or $2x_{\text L}-y_{\text L}+z_{\text L}-2 = \dfrac{4}{3} - \dfrac{1}{6} +\dfrac{5}{6}-2=0$,
donc $\text L \in \text{(BGI)}$.
\end{list}

Le point L est donc le point d'intersection de la droite $d$ et du plan (BGI).
\end{enumerate}

\item %On rappelle que le volume $V$ d'une pyramide est donné par la formule
%\[V=\dfrac{1}{3}\times \mathcal{B}\times h\] où $\mathcal{B}$ est l'aire d'une base et $h$ la hauteur associée à cette base.

\begin{enumerate}
\item La pyramide FBGI a pour base le triangle rectangle FBG, et pour hauteur IF.

\begin{list}{\textbullet}{}
\item $\text{IF} = \dfrac{1}{2}$
\item Le triangle rectangle FBG a pour aire 
$\dfrac{\text{FG}\times \text{FB}}{2} = \dfrac{1}{2}$.
\end{list}

Le volume de la pyramide FBGI est donc $\mathcal{V}=\dfrac{1}{3}\times \dfrac{1}{2}\times \dfrac{1}{2} =\dfrac{1}{12}$.

\item %En déduire l'aire du triangle BGI.
La droite $d$ est orthogonale au plan (BGI) et coupe ce plan en L. Le point F appartient à la droite $d$, donc on peut dire que la distance FL est la distance du point F au plan (BGI), autrement dit c'est la hauteur de la pyramide FBGI dont le triangle BGI est la base.

$\text{FL}^2 = \left (\dfrac{2}{3} -1\right )^2 + \left (\dfrac{1}{6}-0 \right )^2 + \left (\dfrac{5}{6} -1\right )^2
= \dfrac{1}{9} + \dfrac{1}{36} + \dfrac{1}{36} = \dfrac{6}{36} = \dfrac{1}{6}$
donc
$\text{FL} = \dfrac{1}{\sqrt{6}}$

On appelle $\mathcal{A}$ l'aire du triangle BGI.
On exprime le volume de la pyramide FBGI:

$\mathcal{V} = \dfrac{1}{3}\times \text{FL}\times \mathcal{A}
\iff
\dfrac{1}{12} = \dfrac{1}{3}\times \dfrac{1}{\sqrt{6}} \times \mathcal{A}
\iff
\dfrac{3\times \sqrt{6}}{12} = \mathcal{A}
\iff
\mathcal{A} = \dfrac{\sqrt{6}}{4}$

L'aire du triangle BGI est égale à $ \dfrac{\sqrt{6}}{4}$.

\end{enumerate}

\end{enumerate}

\bigskip

