\textbf{\large\textsc{Exercice B} \hfill exercice au choix \hfill 5 points}  

\medskip

\framebox{
\begin{minipage}[]{5.25cm}
\textbf{Principaux domaines abordés}\\
\hspace{2em}Équations différentielles\\
\hspace{2em}Fonction exponentielle ; suites
\end{minipage}
}

\medskip

Dans une boulangerie, les baguettes sortent du four à une température de 225 \textcelsius.

On s'intéresse à l'évolution de la température d'une baguette après sa sortie du four.

On admet qu'on peut modéliser cette évolution à l'aide d'une fonction $f$ définie et dérivable sur l'intervalle $[0~;~+\infty[$.

 Dans cette modélisation, $f(t)$ représente la température en degré Celsius de la baguette au bout de la durée $t$, exprimée en heure, après la sortie du four.
 
Ainsi, $f(0,5)$ représente la température d'une baguette une demi-heure après la sortie du four.

Dans tout l'exercice, la température ambiante de la boulangerie est maintenue à 25 \textcelsius.

On admet alors que la fonction $f$ est solution de l'équation différentielle $y'+ 6y = 150$.

\medskip

\begin{enumerate}
\item 
	\begin{enumerate}
		\item% Préciser la valeur de $f(0)$.
$f(0)$ représente la température d'une baguette lors de sa sortie du four, c'est-à-dire 225 $^{\circ}$C.		
		\item %Résoudre l'équation différentielle $y'+6y = 150$.
Pour résoudre l'équation, on la met sous la forme $y'=ay+b$ avec $a$ et $b$ des réels. On obtient :
        \begin{equation*}
            y'=-6y+150 \iff y'=ay+b \ \text{ avec } \ \begin{cases}
            a=-6 \\ b=150
            \end{cases}
        \end{equation*}
        On sait alors que les solutions de cette équation sont toutes les fonctions de la forme : 
        \begin{equation*}
            f(t)= - \dfrac{b}{a}+C\e^{at}, \ C \in \R
        \end{equation*}
 Les solutions de l'équation différentielle sont donc toutes les fonctions de la forme : 
        \begin{align*}
            f(t) &= - \dfrac{150}{-6} + C\e^{-6t} \\
            f(t) &= 25 + C\e^{-6t}
        \end{align*}		
		
		\item% En déduire que pour tout réel $t\geqslant  0$, on a $f(t) = 200 \e^{-6t}+25$.
La solution de l'équation différentielle a été obtenue en question \textbf{b.} Il reste à exploiter la condition initiale $f(t=0)=f(0)= 225$ d'après la valeur trouvée à la question \textbf{a.} La fonction qui satisfait donc le modèle de l'exercice est la solution de l'équation : 
        \begin{align*}
            f(0)= 225 &\Longleftrightarrow C\e^{0} + 25 = 225 \\
            &\Longleftrightarrow C + 25 = 225 \\
            &\Longleftrightarrow C= 200
        \end{align*}
        Donc on a bien, pour tout réel $t \geqslant 0$ : 
\begin{equation*}
    f(t)= 200\e^{-6t}+25
\end{equation*}		
		
	\end{enumerate}
\item Par expérience, on observe que la température d'une baguette sortant du four décroît et tend à se stabiliser à la température ambiante.

%La fonction $f$ fournit-elle un modèle en accord avec ces observations ?

\begin{itemize}
        \item Vérifions d'abord que la fonction $f$ décroît. $f$ est d'abord bien dérivable pour tout réel $t \geqslant 0$ comme composée de fonctions dérivables et : 
        \begin{equation*}
            \text{pour tout réel} \ t \geqslant 0, \ f'(t)=-1200\e^{-6t}
        \end{equation*}
        Or, pour tout réel $t \geqslant 0$ : 
        \begin{equation*}
          \left\{\begin{array}{cc}
                    \e^{-6t} & > 0  \\
                    -1200 & < 0
            \end{array}\right .
            \Longrightarrow f'(t) < 0 \Longrightarrow f \ \text{est bien  décroissante (strictement)}.
        \end{equation*}
        
        \item Pour vérifier que la température tend à se stabiliser à la température ambiante ($25$ $^{\circ}$C), nous allons calculer la limite de la fonction $f$ en $+ \infty$ : 
        \begin{equation*}
            \lim\limits_{t \rightarrow + \infty}\e^{-6t} = 0 \underset{\text{par produit}}{\Longrightarrow} \lim\limits_{t \rightarrow + \infty} 200\e^{-6t} = 0 \underset{\text{par somme}}{\Longrightarrow} \lim\limits_{t \rightarrow + \infty} 200\e^{-6t} + 25 = 25 = \lim\limits_{t \rightarrow + \infty} f(t). 
        \end{equation*}
        La fonction $f$, qui représente la température de la baguette (en $^{\circ}$C) au bout du temps, a pour limite $25$ en $+\infty$. Cela signifie donc bien que la température tend à se stabiliser à la température ambiante de $25$ $^{\circ}$C.
    \end{itemize}
    
Donc la fonction $f$ fournit un modèle en accord avec ces observations.
    
\item %Montrer que l'équation $f(t) = 40$ admet une unique solution dans $[0~;~+\infty[ $.

 La fonction $f$ est continue et décroissante strictement donc monotone sur $[0:+\infty[$. Par ailleurs, $f(0)=225$ et $\lim\limits_{t \rightarrow + \infty} f(t)=25$ donc, d'après le théorème des valeurs intermédiaires, il existe un unique élément $c \in [0;+\infty[$ tel que $f(c)=40$.
 
Pour mettre les baguettes en rayon, le boulanger attend que leur température soit inférieure ou égale à 40 \textcelsius. On note $\mathcal{T}_0$ le temps d'attente minimal entre la sortie du four d'une baguette et sa mise en rayon.

On donne la représentation graphique de la fonction $f$ dans un repère orthogonal.% ci-dessous

\begin{center}
\psset{xunit=5cm,yunit=0.025cm,labelFontSize=\scriptstyle,comma=true,labelsep=0.1pt}
\begin{pspicture}(-0.4,-20)(2.50,260)
 \multido{\n=-0.1+0.1}{23}{\psline[linewidth=0.3pt,linecolor=lightgray](\n,-20)(\n,250)}
 \multido{\n=-20+20}{14}{\psline[linewidth=0.3pt,linecolor=lightgray](-0.1,\n)(2.1,\n)}
\psaxes[linewidth=0.95pt,Dx=0.5,Dy=20]{->}(0,0)(-0.05,-20)(2.1,250)
\psplot[linewidth=0.85pt,linecolor=blue,plotpoints=5000]{0}{2.1}{2.71828 x 6 mul neg   exp 200 mul 25 add}
\uput[d](1.75,-15){\footnotesize Durée en heure}
\uput[r](0,230){\footnotesize Température en degré Celsius}
\uput[ur](0.3,60){\blue $\mathcal{C}_f$}
\psline[linestyle=dashed,linecolor=red](2,40)(0,40)
\psline[linestyle=dashed,linecolor=red](0.43,40)(0.43,0)
\uput*{5pt}[d](0.43,0){\red \small $~~0,43~~$}
\end{pspicture}
\end{center}

\item %Avec la précision permise par le graphique, lire $\mathcal{T}_0$. On donnera une valeur approchée de $\mathcal{T}_0$ sous forme d'un nombre entier de minutes.
 La courbe $\mathcal{C}_f$ semble atteindre $40$ vers $0,43$ heure soit $0,43 \times 60 = 25,8$ minutes. On trouve donc une valeur approchée de $26$ minutes.

\item On s'intéresse ici à la diminution, minute après minute, de la température d'une baguette à sa sortie du four.

Ainsi, pour un entier naturel $n$, ${D}_n$ désigne la diminution de température en degré Celsius d'une baguette entre la $n$-ième et la $(n+1)$-ième minute après sa sortie du four.

On admet que, pour tout entier naturel $n$ : 
$D_n=f\left(\dfrac{n}{60}\right)-f\left(\dfrac{n+1}{60}\right)$.

%\smallskip

	\begin{enumerate}
		\item% Vérifier que 19 est une valeur approchée de $D_0$ à 0,1 près, et interpréter ce résultat dans le contexte de l'exercice.
On cherche une valeur approchée de $D_0$.		
		
\begin{align*}
{D}_0 &= f\left(\dfrac{0}{60}\right)- f\left(\dfrac{1}{60}\right) \\
&= f(0) - f\left(\dfrac{1}{60}\right) \\
&= 200\e^0 + \cancel{25} - \left(200\e^{-\frac{6}{60}} + \cancel{25}\right) \\
&= 200 - 200\e^{-\frac{6}{60}} \\
&\approx 19,03
\end{align*}
Donc $19$ est bien une valeur approchée de $\mathcal{D}_0$ à $0,1$ près. Cela signifie que la diminution de température qui se fait lors de la première minute après la sortie du four est d'environ $19$ $^{\circ}$C. Au bout d'une minute, la baguette est donc à $225-19 = 206 \ ^{\circ}$C.
		
%Vérifier que l'on a, pour tout entier naturel $n$:  ${D}_n=200 \e^{-0,1n}(1 - \e^{-0,1}) .$

%En déduire le sens de variation de la suite $\left({D}_n\right)$, puis la limite de la suite $\left({D}_n\right)$.

%Ce résultat était-il prévisible dans le contexte de l'exercice ?

\item \begin{align*}
{D}_n &=f\left(\dfrac{n}{60}\right)- f\left(\dfrac{n+1}{60}\right) \\
&= 200\e^{- 6 \times \frac{n}{60}} + \cancel{25} - \left(200\e^{-6 \times \frac{n+1}{60}} + \cancel{25}     \right) \\
&= 200\e^{-0,1n} - 200\e^{\frac{-6n-6}{60}} \\
&= 200\e^{-0,1n} - 200\e^{\frac{-6n}{60} + \left(\frac{-6}{60}\right)} \\
&= 200\e^{-0,1n} - 200\e^{-0,1n} \times \e^{-0,1} \\
\mathcal{D}_n &= 200\e^{-0,1n}\left(1-\e^{-0,1}\right)
\end{align*}
Pour étudier le sens de variation de la suite $({D}_n)$, on étudie le signe de ${D}_{n+1}-{D}_n$. \\
Pour tout entier naturel $n$ : 
\begin{align*}
{D}_{n+1}-{D}_n &= 200\e^{-0,1(n+1)}\left(1-\e^{-0,1}\right) - 200\e^{-0,1n}\left(1-\e^{-0,1}\right) \\
&= 200\e^{-0,1n} \times \e^{-0,1} \left(1-\e^{-0,1}\right)- 200\e^{-0,1n}\left(1-\e^{-0,1}\right) \\
{D}_{n+1}-{D}_n &= 200\e^{-0,1n}\left(1-\e^{-0,1}\right) \left[\e^{-0,1}-1\right]
\end{align*}
Étudions le signe de cette expression pour tout entier naturel $n$ : 
\begin{equation*}
\left\{\begin{array}{cc}
200\e^{-0,1n} &\geqslant 0  \\
1-\e^{-0,1} &\geqslant 0 \\
\e^{-0,1}-1 &\leqslant 0
\end{array}\right .
\underset{\text{par produit}}{\Longrightarrow} \mathcal{D}_{n+1}-\mathcal{D}_n \leqslant 0 \Longrightarrow \text{la suite} \ (\mathcal{D}_n) \ \text{est décroissante}.
 \end{equation*}
 Calculons alors la limite de cette suite : 
 \begin{equation*}
\left\{\begin{array}{cl}
\lim\limits_{n \rightarrow + \infty} 200\e^{-0,1n} &= 0  \\
\lim\limits_{n \rightarrow + \infty} 1-\e^{-0,1} &= 1-\e^{-0,1} \\
\lim\limits_{n \rightarrow + \infty} \e^{-0,1}-1 &= \e^{-0,1} -1
\end{array} \right .
\underset{\text{par produit}}{\Longrightarrow} \lim\limits_{n \rightarrow + \infty} {D}_n=0
\end{equation*}
Nous trouvons une limite de $0$ pour ${D}_n$. Puisque la baguette tend à se stabiliser à la température ambiante, la diminution de température entre la $n$-ième  et la $(n+1)$-ième minute va tendre vers $0$. Le résultat était bien prévisible dans le contexte de l'exercice. 
		\end{enumerate}
\end{enumerate}
