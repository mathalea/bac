
\medskip

Dans le parc national des Pyrénées, un chercheur travaille sur le déclin d'une espèce protégée dans les lacs de haute-montagne : le \og crapaud accoucheur \fg.

Les parties I et II peuvent être abordées de façon indépendante.

\begin{center}\textbf{Partie I : Effet de l'introduction d'une nouvelle espèce.}\end{center}

Dans certains lacs des Pyrénées, des truites ont été introduites par l'homme afin de permettre des activités de pêche en montagne. Le chercheur a étudié l'impact de cette introduction sur la
population de crapauds accoucheurs d'un lac.

Ses études précédentes l'amènent à modéliser l'évolution de cette population en fonction du temps par la fonction $f$ suivante : 

\[f(t) = \left(0,04t^2 - 8t + 400\right)\text{e}^{\frac{t}{50}} + 40 \, \text{ pour }\,  t \in [0~;~120]\]

La variable $t$ représente le temps écoulé, en jour, à partir de l'introduction à l'instant $t = 0$ des truites dans le lac, et $f(t)$ modélise le nombre de crapauds à l'instant $t$.

\medskip

\begin{enumerate}
\item Déterminer le nombre de crapauds présents dans le lac lors de l'introduction des truites.
\item On admet que la fonction $f$ est dérivable sur l'intervalle [0~;~120] et on note $f'$ sa fonction dérivée.

Montrer, en faisant apparaitre les étapes du calcul, que pour tout nombre réel $t$ appartenant à 
l'intervalle [0~;~120] on a : 

\[f'(t) = t(t - 100)\text{e}^{\frac{t}{50}} \times 8 \times 10^{-4}.\]

\item Étudier les variations de la fonction $f$ sur l'intervalle [0~;~120], puis dresser le tableau de variations de $f$ sur cet intervalle (on donnera des valeurs approchées au centième).
\item Selon cette modélisation:
	\begin{enumerate}
		\item Déterminer le nombre de jours $J$ nécessaires afin que le nombre de crapauds atteigne son minimum. Quel est ce nombre minimum ?
		\item Justifier que, après avoir atteint son minimum, le nombre de crapauds dépassera un jour $140$ individus.
		\item À l'aide de la calculatrice, déterminer la durée en jour à partir de laquelle le nombre de crapauds dépassera $140$ individus.
	\end{enumerate}
\end{enumerate}

\begin{center}\textbf{Partie II : Effet de la Chytridiomycose sur une population de têtards}\end{center}

Une des principales causes du déclin de cette espèce de crapaud en haute montagne est une maladie, la \og Chytridiomycose \fg, provoquée par un champignon.

Le chercheur considère que :

\begin{itemize}
\item[$\bullet~~$]Les trois quarts des lacs de montagne des Pyrénées ne sont pas infectés par le champignon, c'est-à-dire qu'ils ne contiennent aucun têtard (larve du crapaud) contaminé.
\item[$\bullet~~$]Dans les lacs restants, la probabilité qu'un têtard soit contaminé est de $0,74$.
\end{itemize}

Le chercheur choisit au hasard un lac des Pyrénées, et y procède à des prélèvements.

\emph{Pour la suite de l'exercice, les résultats seront arrondis au millième lorsque cela est nécessaire.}

Le chercheur prélève au hasard un têtard du lac choisi afin d'effectuer un test avant de le relâcher.
 
On notera $T$ l'évènement \og Le têtard est contaminé par la maladie\fg{} et $L$ l'évènement \og Le lac est infecté par le champignon \fg.

On notera $\overline{L}$ l'évènement contraire de $L$ et $\overline{T}$ l'évènement contraire de $T$.

\medskip

\begin{enumerate}
\item Recopier et compléter l'arbre de probabilité suivant en utilisant les données de l'énoncé:

\begin{center}
\pstree[treemode=R,nodesepB=3pt,levelsep=2.75cm]{\TR{}}
{\pstree{\TR{$L$~~}}
	{\TR{$T$}
	\TR{$\overline{T}$}
	}
\pstree{\TR{$\overline{L}$~~}}
	{\TR{$T$}\taput{0}
	\TR{$\overline{T}$}\tbput{1}
	}	
}
\end{center}

\item Montrer que la probabilité $P(T)$ que le têtard prélevé soit contaminé est de $0,185$.
\item Le têtard n'est pas contaminé. Quelle est la probabilité que le lac soit infecté ?
\end{enumerate}

\bigskip

