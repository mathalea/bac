
\textbf{Commun à tous les candidats}

\medskip

Au 1\up{er} janvier 2020, la centrale solaire de Big Sun possédait \np{10560} panneaux solaires.

On observe, chaque année, que 2\,\% des panneaux se sont détériorés et nécessitent d'être retirés tandis que 250 nouveaux panneaux solaires sont installés.

\bigskip

\textbf{Partie A - Modélisation à l'aide d'une suite}

\medskip

On modélise l'évolution du nombre de panneaux solaires par la suite $\left(u_n\right)$ définie par $u_0 = \np{10560}$ et, pour tout entier naturel $n$,\, $u_{n+1} = 0,98u_n +250$, où $u_n$  est le nombre de panneaux solaires au 1\up{er} janvier de l'année $2020 +n$.

\medskip

\begin{enumerate}
\item 
	\begin{enumerate}
		\item  Expliquer en quoi cette modélisation correspond à la situation étudiée.
		\item  On souhaite savoir au bout de combien d'années le nombre de panneaux solaires sera strictement supérieur à \np{12000}.
		
À l'aide de la calculatrice, donner la réponse à ce
problème.
		\item  Recopier et compléter le programme en Python ci-dessous de sorte que la valeur
cherchée à la question précédente soit stockée dans la variable n à l'issue de l'exécution de ce dernier.
\begin{center}
\begin{tabular}{|l|}\hline
u $= \np{10560}$\\
n$ =0$\\
while  \ldots \ldots :\\
\qquad u $= \ldots \ldots$ \\
\qquad n $= \ldots \ldots$\\ \hline
\end{tabular}
\end{center}

	\end{enumerate}
\item Démontrer par récurrence que, pour tout entier naturel $n$, on a $u_n \leqslant \np{12500}$.
\item Démontrer que la suite $\left(u_n\right)$ est croissante.
\item En déduire que la suite $\left(u_n\right)$ converge. Il n'est pas demandé, ici, de calculer sa limite.
\item On définit la suite $\left(v_n\right)$  par $v_n = u_n - \np{12500}$, pour tout entier naturel $n$.
	\begin{enumerate}
		\item Démontrer que la suite $\left(v_n\right)$ est une suite géométrique de raison $0,98$ dont on précisera le premier terme.
		\item Exprimer, pour tout entier naturel $n$,\, $v_n$ en fonction de $n$.
		\item En déduire, pour tout entier naturel $n$,\, $u_n$ en fonction de $n$.
		\item Déterminer la limite de la suite $\left(u_n\right)$. 
		
Interpréter ce résultat dans le contexte du modèle.
	\end{enumerate}
\end{enumerate}

\bigskip

\textbf{Partie B - Modélisation à l'aide d'une fonction}

\medskip

Une modélisation plus précise a permis d'estimer le nombre de panneaux solaires de la centrale à l'aide de la fonction $f$ définie pour tout $x \in [0~;~ +\infty[$ par 

\[f(x) = \np{12500} - 500 \e^{-0,02x+1,4},\]

 où $x$ représente le nombre d'années écoulées depuis le 1\up{er} janvier 2020.

\medskip

\begin{enumerate}
\item Étudier le sens de variation de la fonction $f$.
\item Déterminer la limite de la fonction $f$ en $+\infty$.
\item En utilisant ce modèle, déterminer au bout de combien d'années le nombre de panneaux
solaires dépassera \np{12000}.
\end{enumerate}

\bigskip

