
\textbf{Commun à tous les candidats}

\medskip

On considère la suite $\left(u_n\right)$ définie par : $u_0 = 1$ et, pour tout entier naturel $n$,

\[ u_{n+1}  = \dfrac{4u_n}{u_n + 4}.\]

\smallskip

\begin{enumerate}
\item ~

\parbox{0.7\linewidth}{
La copie d'écran ci-contre présente les valeurs, calculées à l'aide d'un tableur, des
termes de la suite $\left(u_n\right)$ pour $n$ variant de 0 à 12, ainsi que celles du quotient 
$\dfrac{4}{u_n}$,  (avec, pour les valeurs de $u_n$, affichage de deux
chiffres pour les parties décimales).

À l'aide de ces valeurs, conjecturer l'expression de $\dfrac{4}{u_n}$  en fonction de $n$.

Le but de cet exercice est de démontrer cette conjecture (question \textbf{5.}), et d'en déduire la limite de la suite $\left(u_n\right)$ (question \textbf{6.}).}\hfill
\parbox{0.25\linewidth}{
$\begin{array}{|*{3}{c|}}\hline
n&u_n&\dfrac{4}{u_n}\\ \hline
0 &1,00 &4\\ \hline
1 &0,80 &5\\ \hline
2 &0,67 &6\\ \hline
3 &0,57 &7\\ \hline
4 &0,50 &8\\ \hline
5 &0,44 &9\\ \hline
6 &0,40 &10\\ \hline
7 &0,36 &11\\ \hline
8 &0,33 &12\\ \hline
9 &0,31 &13\\ \hline
10 &0,29& 14 \\ \hline
11 &0,27& 15 \\ \hline
12 &0,25 &16\\ \hline
\end{array}$}

\item Démontrer par récurrence que, pour tout entier naturel $n$, on a : $u_n > 0$.
\item Démontrer que la suite $\left(u_n\right)$ est décroissante.
\item Que peut-on conclure des questions \textbf{2.} et \textbf{3.} concernant la suite $\left(u_n\right)$ ?
\item On considère la suite $\left(v_n\right)$ définie pour tout entier naturel $n$ par : $v_n = \dfrac{4}{u_n}$.

Démontrer que $\left(v_n\right)$ est une suite arithmétique. 

Préciser sa raison et son premier terme. 

En déduire, pour tout entier naturel $n$, l'expression de $v_n$ en fonction de $n$.
\item Déterminer, pour tout entier naturel $n$, l'expression de $u_n$ en fonction de $n$.

En déduire la limite de la suite $\left(u_n\right)$.
\end{enumerate}

\bigskip

