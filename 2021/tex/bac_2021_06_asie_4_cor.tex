\textbf{EXERCICE -- A}

\medskip

\begin{tabular}{|l|}\hline
\textbf{Principaux domaines abordés}\\
-- convexité\\
-- fonction logarithme\\ \hline
\end{tabular}

\bigskip

\textbf{Partie I : lectures graphiques}

\medskip

$f$ désigne une fonction définie et dérivable sur $\R$.

On donne ci-dessous la courbe représentative de la fonction dérivée $f'$.

\begin{center}
\psset{xunit=0.9cm,yunit=3cm}
\begin{pspicture}(-7,-0.8)(7,1.2)
\multido{\n=-7+1}{15}{\psline[linestyle=dashed,linewidth=0.1pt](\n,-0.8)(\n,1.2)}
\multido{\n=-0.8+0.2}{11}{\psline[linestyle=dashed,linewidth=0.1pt](-7,\n)(7,\n)}
\psaxes[linewidth=1.25pt,labelFontSize=\scriptstyle]{->}(0,0)(-7,-0.8)(7,1.2)
\psplot[plotpoints=2000,linewidth=1.25pt,linecolor=red]{-7}{7}{x 2 mul 1 add x dup mul x add 2.5 add div}
\uput[u](0,1){\red Courbe de la fonction dérivée $f'$}
\end{pspicture}
\end{center}

\medskip

%\emph{Avec la précision permise par le graphique, répondre aux questions suivantes}

\medskip

\begin{enumerate}
\item %Déterminer le coefficient directeur de la tangente à la courbe de la fonction $f$ en 0.
On lit $f'(0) = 0,4 = \dfrac{2}{5}$. 
\item 
	\begin{enumerate}
		\item %Donner les variations de la fonction dérivée $f'$.
D'après la figure :
		
\starredbullet~$f'(x)$ est croissante si  $x \in [-2~;~1]$;

\starredbullet~$f'(x)$ est décroissante si $x < - 2$ et si $x > 1$.


		\item %En déduire un intervalle sur lequel $f$ est convexe.
\starredbullet~$f'\left(- \dfrac{1}{2}\right) = 0$.

Donc $f''(x) > 0$ sur l'intervalle $[-2~;~1]$ ; la fonction $f$ est convexe sur l'intervalle $[-2~;~1]$.
	\end{enumerate}
\end{enumerate}

\bigskip

\textbf{Partie II : étude de fonction}

\medskip

La fonction $f$ est définie sur $\R$ par 

\[f(x) = \ln \left(x^2 + x + \dfrac{5}{2}\right).\]

\medskip

\begin{enumerate}
\item %Calculer les limites de la fonction $f$ en $+\infty$ et en $-\infty$.
\starredbullet~On a $\displaystyle\lim_{x \to + \infty} x^2 + x + \dfrac{5}{2} = + \infty$, d'où par composition de limites $\displaystyle\lim_{x \to + \infty} f(x) = + \infty$.

\starredbullet~On a $x^2 + x + \dfrac{5}{2} = x^2\left(1 + \dfrac{1}{x} + \dfrac{5}{2x^2}\right)$.

Donc $f(x) = \ln x^2\left(1 + \dfrac{1}{x} + \dfrac{5}{2x^2}\right) = \ln x^2 + \ln \left(1 + \dfrac{1}{x} + \dfrac{5}{2x^2}\right)$.

Or $\displaystyle\lim_{x \to - \infty}\dfrac{1}{x} = \displaystyle\lim_{x \to - \infty}\dfrac{5}{2x^2x} = 0$, donc 
$\displaystyle\lim_{x \to - \infty}1 + \dfrac{1}{x} + \dfrac{5}{2x^2} = 1$ et $\displaystyle\lim_{x \to - \infty}\ln \left(1 + \dfrac{1}{x} + \dfrac{5}{2x^2}\right) = \ln 1 = 0$.

Finalement :

$\displaystyle\lim_{x \to - \infty}f(x) = \displaystyle\lim_{x \to - \infty}\ln x^2 = + \infty$.
\item %Déterminer une expression $f'(x)$ de la fonction dérivée de $f$ pour tout $x \in \R$.
On a $f(x) = \ln u(x)$, avec $u(x) = x^2 + x + \dfrac{5}{2}$.

$u$ étant dérivable sur $\R$ et pour le trinôme $x^2 + x + \dfrac{5}{2}$, \, $\Delta = 1 - 10 = - 9 < 0$, donc 

$x^2 + x + \dfrac{5}{2} > 0$ quel que soit le réel $x$.

La fonction $\ln u$ est donc dérivable sur $\R$ et sur cet intervalle :

$\left(\ln u \right)' = \dfrac{u'(x)}{u(x)} = \dfrac{2x + 1}{x^2 + x + \dfrac{5}{2}}$.

Conclusion : quel que soit $x \in \R$, \, $f'(x) = \dfrac{2x + 1}{x^2 + x + \dfrac{5}{2}}$.
\item %En déduire le tableau des variations de $f$. On veillera à placer les limites dans ce tableau.
On a vu que $x^2 + x + \dfrac{5}{2} > 0$ sur $\R$ ; le signe de $f'(x)$ est donc celui de $2x + 1$ :

\starredbullet $f'(x) > 0 \iff 2x + 1 > 0 \iff x > - \dfrac{1}{2}$ : la fonction $f$ est croissante sur $\left]- \dfrac{1}{2}~;~+ \infty\right[$ ;

\starredbullet $f'(x) < 0 \iff 2x + 1 < 0 \iff x < - \dfrac{1}{2}$ : la fonction $f$ est décroissante sur $\left]- \infty~;~-\dfrac{1}{2} \right[$. 

On a $f\left(-\dfrac{1}{2}\right) = \ln \left(\left(- \dfrac{1}{2}\right)^2 - \dfrac{1}{2} + \dfrac{5}{2} \right) = \ln \left(\dfrac{1}{4} - \dfrac{1}{2} + \dfrac{5}{2}\right) = \ln \dfrac{9}{4}$.

D'où le tableau de variations de $f$ :

\begin{center}
\psset{unit=1cm}
\begin{pspicture}(7,3)
\psframe(7,3)\psline(0,2)(7,2)\psline(0,2.5)(7,2.5)\psline(1,0)(1,3)
\uput[u](0.5,2.4){$x$} \uput[u](1.5,2.4){$- \infty$} \uput[u](4,2.4){$- \frac{1}{2}$} \uput[u](6.5,2.4){$+ \infty$} 
\uput[d](1.5,2){$+ \infty$}\uput[u](4,0){$\ln \dfrac{9}{4}$}\uput[d](6.5,2){$+ \infty$}
\rput(0.5,1){$f$}
\psline{->}(1.5,1.5)(3.5,0.5)\psline{->}(4.5,0.5)(6.5,1.5)
\end{pspicture}
\end{center}
\item 
	\begin{enumerate}
		\item %Justifier que l'équation $f(x) = 2$ a une unique solution $\alpha$ dans l'intervalle $\left[-\dfrac{1}{2}~;~+ \infty\right[$.
		
Dans la tableau précédent $f\left(-\dfrac{1}{2}\right) =\ln \dfrac{9}{4} \approx 0,81$.

Sur l'intervalle $\left[-\dfrac{1}{2}~;~+ \infty\right[$ la fonction $f$ est continue car dérivable et comme $2 \in \left[\ln \frac{9}{4}~;~+ \infty\right[$, il existe d'après le théorème des valeurs intermédiaires un réel unique $\alpha \in \left[-\dfrac{1}{2}~;~+ \infty\right[$ tel que $f(\alpha) = 2$.
		\item %Donner une valeur approchée de $\alpha$ à $10^{-1}$ près.
La calculatrice donne :

$f(1) \approx 1,5$ et $f(2) \approx 2,14$, donc $\alpha \in ]1~;~2[$ ;

$f(1,7) \approx 1,96$ et $f(1,8) \approx 2,02$, donc $\alpha \in ]1,7~;~1,8[$ ;

$f(1,76) \approx 1,995$ et $f(1,77) \approx 2,002$, donc $\alpha \in ]1,76~;~1,77[$.

Conclusion $\alpha \approx 1,8$ à $10^{-1}$ près.
 	\end{enumerate}
\item ~%La fonction $f'$ est dérivable sur $\R$. On admet que, pour tout $x \in  \R$,\,  $f''(x) =
%$\dfrac{-2x^2 - 2x + 4}{\left(x^2 + x + \dfrac{5}{2}\right)^2}$.

%Déterminer le nombre de points d'inflexion de la courbe représentative de $f$.

La fonction a un point d'inflexion si en ce point sa dérivée seconde s'annule en changeant de signe.

On a $f'(x) = \dfrac{2x + 1}{x^2 + x + \dfrac52}$ : cette fonction est dérivable car le dénominateur ne s'annule pas, donc sur $\R$ :

$f''(x) = \dfrac{2\left(x^2 + x +  \frac52\right) - (2x + 1)(2x + 1)}{\left(x^2 + x + \frac52\right)^2} =
\dfrac{2x^2 + 2x + 5 - 4x^2 - 4x - 1}{\left(x^2 + x + \frac52\right)^2} =$

$\dfrac{- 2x^2 - 2x + 4}{\left(x^2 + x + \frac52\right)^2} = \dfrac{-2\left(x^2 + x - 2 \right)}{\left(x^2 + x + \frac52\right)^2} ;
f''(x) = \dfrac{-2\left[\left( x + \frac12\right) - \frac14 - 2 \right]}{\left(x^2 + x + \frac52\right)^2} = $

$\dfrac{- 2\left[\left(x + \frac12\right)^2 - \frac94 \right]}{\left(x^2 + x + \frac52\right)^2} =\dfrac{- 2\left(x + \frac12 + \frac32 \right)\left(x + \frac12 - \frac32 \right)}{\left(x^2 + x + \frac52\right)^2} =
\dfrac{-2(x + 2)(x - 1)}{\left(x^2 + x + \frac52\right)^2}$.

Comme $\left(x^2 + x + \frac{5}{2}\right)^2 > 0$, quel que soit le réel $x$, le signe de $f''(x)$ est celui du numérateur $-2(x + 2)(x - 1) = 2(x + 2)(1 - x)$, soit celui du trinôme $(x + 2)(1 - x)$. On en déduit le 

Tableau de signes :

\begin{center}
\psset{unit=1cm}
\begin{pspicture}(-1,0)(7,2)
\psframe(-1,0)(7,2)\psline(-1,0.5)(7,0.5)\psline(-1,1)(7,1)\psline(-1,1.5)(7,1.5)\psline(1,0)(1,2)
\psline(-1,0)(7,0)
\uput[u](0,1.4){$x$} \uput[u](1.5,1.4){$-\infty$} \uput[u](3,1.4){$-2$} \uput[u](5,1.4){$1$} \uput[u](6.5,1.4){$+\infty$} 
\uput[u](0,0.9){$x+2$} \uput[u](2,0.9){$-$} \uput[u](3,0.9){$0$}  \uput[u](4,0.9){$+$}\uput[u](6,0.9){$+$}
\uput[u](0,0.4){$1 - x$} \uput[u](2,0.4){$+$} \uput[u](4,0.4){$+$}\uput[u](5,0.4){$0$} \uput[u](6,0.4){$-$} 
\uput[u](0,-0.1){\small $(x+2)(1 - x)$} \uput[u](2,-0.1){$-$} \uput[u](3,-0.1){$0$} \uput[u](4,-0.1){$+$}\uput[u](5,-0.1){$0$}\uput[u](6,-0.1){$-$}
%\uput[u](0,-0.6){\small $-(x+1)(x - 2)$}\uput[u](2,-0.6){$-$}\uput[u](3,-0.6){$0$}\uput[u](4,-0.6){$+$}\uput[u](5,-0.6){$0$}\uput[u](6,-0.6){$-$}
\end{pspicture}
\end{center}
\end{enumerate}


%$$\sqrt{y} + \sqrt{x} + \sqrt{\smash[b]{y}}$$

On constate que la dérivée seconde s'annule en changeant de signe en $- 2$ et en 1. La courbe a donc deux points d'inflexion.

\bigskip

