
\smallskip

\begin{tabular}{|l|}\hline
Principaux domaines abordés :\\
Fonction logarithme; dérivation\\ \hline
\end{tabular}

\begin{center}\textbf{Partie 1}\end{center}

On désigne par $h$ la fonction définie sur l'intervalle $]0~;~ +\infty[$ par:

\[h(x) = 1 + \dfrac{\ln (x)}{x^2}.\]

On admet que la fonction $h$ est dérivable sur $]0~;~ +\infty[$ et on note $h'$ sa fonction dérivée.

\medskip

\begin{enumerate}
\item Déterminez les limites de $h$ en $0$ et en $+ \infty$.
\item Montrer que, pour tout nombre réel $x$ de $]0~;~ +\infty[$,\, $h'(x) = \dfrac{1 - 2|n (x)}{x^3}$.
\item En déduire les variations de la fonction $h$ sur l'intervalle $]0~;~ +\infty[$.
\item Montrer que l'équation $h(x) = 0$ admet une solution unique $\alpha$ appartenant à $]0~;~ +\infty[$ et vérifier que : $\dfrac{1}{2} < \alpha < 1$.
\item Déterminer le signe de $h(x)$ pour $x$ appartenant à $]0~;~ +\infty[$.
\end{enumerate}

\bigskip

\begin{center}\textbf{Partie 2}\end{center}

On désigne par $f_1$ et $f_2$ les fonctions définies sur $]0~;~ +\infty[$ par :

\[f_1(x) = x-1 - \dfrac{\ln (x)}{x^2}\qquad \text{et}\qquad f_2(x) = x - 2 - \dfrac{2\ln (x)}{x^2}.\]

On note $\mathcal{C}_1$ et $\mathcal{C}_2$ les représentations graphiques respectives de $f_1$ et $f_2$ dans un repère \Oij.

\medskip

\begin{enumerate}
\item Montrer que, pour tout nombre réel $x$ appartenant à $]0~;~ +\infty[$, on a : 

\[f_1(x) - f_2(x) = h(x).\]

\item Déduire des résultats de la Partie 1 la position relative des courbes $\mathcal{C}_1$ et $\mathcal{C}_2$.

On justifiera que leur unique point d'intersection a pour coordonnées $(\alpha~;~\alpha)$.

On rappelle que $\alpha$ est l'unique solution de l'équation $h(x) = 0$.
\end{enumerate}

\bigskip

