
\textbf{Commun à tous les candidats}

\medskip
Une société de jeu en ligne propose une nouvelle application pour smartphone nommée \og Tickets coeurs! \fg.

Chaque participant génère sur son smartphone un ticket comportant une grille de taille $3 \times 3$ sur laquelle sont placés trois cœurs répartis au hasard, comme par exemple ci-dessous.

\begin{center}
\psset{unit=1cm}
\begin{pspicture}(3,3)
\multido{\n=0+1}{4}{\psline(\n,0)(\n,3)}
\multido{\n=0+1}{4}{\psline(0,\n)(3,\n)}
\rput(1.5,2.5){$\heartsuit$}\rput(0.5,1.5){$\heartsuit$}\rput(2.5,0.5){$\heartsuit$}
\end{pspicture}
\end{center}

Le ticket est gagnant si les trois cœurs sont positionnés côte à côte sur une même ligne, sur une
même colonne ou sur une même diagonale.

\medskip

\begin{enumerate}
\item Justifier qu'il y a exactement $84$ façons différentes de positionner les trois cœurs sur une grille.
\item Montrer que la probabilité qu'un ticket soit gagnant est égale à $\dfrac{2}{21}$.
\item Lorsqu'un joueur génère un ticket, la société prélève 1~\euro{} sur son compte en banque. Si le ticket est gagnant, la société verse alors au joueur $5$~\euro. Le jeu est-il favorable au joueur?
\item Un joueur décide de générer $20$ tickets sur cette application. On suppose que les générations des tickets sont indépendantes entre elles.
	\begin{enumerate}
		\item Donner la loi de probabilité de la variable aléatoire $X$ qui compte le nombre de tickets gagnants parmi les $20$ tickets générés.
		\item Calculer la probabilité, arrondie à $10^{-3}$, de l'évènement $(X  =  5)$.
		\item Calculer la probabilité, arrondie à $10^{-3}$, de l'évènement $(X \geqslant 1)$ et interpréter le résultat dans le contexte de l'exercice.
	\end{enumerate}
\end{enumerate}

\bigskip

