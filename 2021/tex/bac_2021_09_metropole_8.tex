
\textbf{Commun à tous les candidats}

\medskip

\begin{center}\textbf{Partie I}\end{center}

On considère la fonction $f$ définie sur $\R$ par 

\[f(x) = x - \text{e}^{-2x}.\]

On appelle $\Gamma$ la courbe représentative de la fonction $f$ dans un repère orthonormé \Oij.

\medskip

\begin{enumerate}
\item Déterminer les limites de la fonction $f$ en $- \infty$ et en $+ \infty$.
\item Étudier le sens de variation de la fonction $f$ sur $\R$ et dresser son tableau de variation.
\item Montrer que l'équation $f(x) = 0$ admet une unique solution $\alpha$ sur $\R$, dont on donnera une valeur approchée à $10^{-2}$ près.
\item Déduire des questions précédentes le signe de $f(x)$ suivant les valeurs de $x$.
\end{enumerate}

\begin{center}\textbf{Partie II}\end{center}


Dans le repère orthonormé \Oij, on appelle $\mathcal{C}$ la courbe représentative de la fonction $g$ définie sur $\R$ par:

\[g(x) = \text{e}^{-x}.\]

 La courbes $\mathcal{C}$ et la courbe $\Gamma$ (qui représente la fonction $f$ de la Partie I) sont tracées sur le \textbf{graphique donné en annexe qui est à compléter et à rendre avec la copie.}

\smallskip

Le but de cette partie est de déterminer le point de la courbe $\mathcal{C}$ le plus proche de l'origine O du repère et d'étudier la tangente à $\mathcal{C}$ en ce point.

\medskip

\begin{enumerate}
\item Pour tout nombre réel $t$, on note $M$ le point de coordonnées $\left(t~;~\text{e}^{-t}\right)$ de la courbe $\mathcal{C}$.

On considère la fonction $h$ qui, au nombre réel $t$, associe la distance O$M$.

On a donc: $h(t) = \text{O}M$, c'est-à-dire :

\[h(t) = \sqrt{t^2 + \text{e}^{-2t}}\]

	\begin{enumerate}
		\item Montrer que, pour tout nombre réel $t$,

\[h'(t) = \dfrac{f(t)}{\sqrt{t^2 + \text{e}^{-2t}}}.\]

où $f$ désigne la fonction étudiée dans la \textbf{Partie I}.
		\item Démontrer que le point A de coordonnées $\left(\alpha~;~\text{e}^{-\alpha}\right)$ est le point de la courbe $\mathcal{C}$ pour lequel la longueur O$M$ est minimale.
		

Placer ce point sur le \textbf{graphique donné en annexe, à rendre avec la copie}.
	\end{enumerate}
\item On appelle $T$ la tangente en A à la courbe $\mathcal{C}$.
	\begin{enumerate}
		\item Exprimer en fonction de $\alpha$ le coefficient directeur de la tangente $T$.
		
On rappelle que le coefficient directeur de la droite (OA) est égal à $\dfrac{\text{e}^{-\alpha}}{\alpha}$.

On rappelle également le résultat suivant qui pourra être utilisé sans démonstration:

\emph{Dans un repère orthonormé du plan, deux droites $D$ et $D'$ de coefficients directeurs respectifs $m$ et $m'$ sont perpendiculaires si, et seulement si le produit $mm'$ est égal à $-1$.}

		\item Démontrer que la droite (OA) et la tangente $T$ sont perpendiculaires. 
		
Tracer ces droites sur le \textbf{graphique donné en annexe, à rendre avec la copie.}
	\end{enumerate}
\end{enumerate}

\begin{center}

	\textbf{\large Annexe à compléter et à rendre avec la copie}
	

	\psset{unit=3.25cm,comma=true}
	\begin{pspicture*}(-1,-0.75)(3.25,2.6)
	\psgrid[gridlabels=0pt,subgriddiv=2,gridwidth=0.08pt](-1,-1)(4,3)
	\psaxes[linewidth=1.25pt,Dx=0.5,Dy=0.5]{->}(0,0)(-0.99,-0.75)(3.25,2.6)
	\uput[l](1.5,1.5){\red $\Gamma$}\uput[ur](-0.75,2){\blue $\mathcal{C}$}
	\psplot[plotpoints=2000,linecolor=red,linewidth=1.25pt]{-1}{3.25}{x 1 2.71828 2 x mul exp div sub}
	\psplot[plotpoints=2000,linecolor=blue,linewidth=1.25pt]{-1}{3.25}{1 2.71828  x  exp div}
	\end{pspicture*}
	
	\end{center}

\bigskip

