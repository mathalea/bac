
\medskip

\framebox{
\begin{minipage}[]{5.25cm}
\textbf{Principaux domaines abordés}\\
\hspace{2em}Équations différentielles\\
\hspace{2em}Fonction exponentielle ; suites
\end{minipage}
}

\medskip

Dans une boulangerie, les baguettes sortent du four à une température de 225 \textcelsius.

On s’intéresse à l’évolution de la température d’une baguette après sa sortie du four.

On admet qu’on peut modéliser cette évolution à l’aide d’une fonction $f$ définie et dérivable sur l’intervalle $[0~;~+\infty[$.

 Dans cette modélisation, $f(t)$ représente la température en degré Celsius de la baguette au bout de la durée $t$, exprimée en heure, après la sortie du four.

Ainsi,$f(0,5)$ représente la température d’une baguette une demi-heure après la sortie du four.

Dans tout l’exercice, la température ambiante de la boulangerie est maintenue à 25 \textcelsius.

On admet alors que la fonction $f$ est solution de l'équation différentielle $y'+ 6y = 150$.

\medskip

\begin{enumerate}
\item 
	\begin{enumerate}
		\item Préciser la valeur de $f(0)$.
		\item Résoudre l’équation différentielle $y'+6y = 150$.
		\item En déduire que pour tout réel $t\geqslant  0$, on a $f(t) = 200 \e^{-6t}+25$.
	\end{enumerate}
\item Par expérience, on observe que la température d’une baguette sortant du four :

\begin{itemize}
\item  décroît ;
\item tend à se stabiliser à la température ambiante.
\end{itemize}

La fonction $f$ fournit-elle un modèle en accord avec ces observations ?

\item Montrer que l’équation $f(t) = 40$ admet une unique solution dans $[0~;~+\infty[ $.

Pour mettre les baguettes en rayon, le boulanger attend que leur température soit inférieure ou égale à 40 \textcelsius. On note $\mathcal{T}_0$ le temps d’attente minimal entre la sortie du four d’une baguette et sa mise en rayon.

On donne en page suivante la représentation graphique de la fonction $f $dans un repère orthogonal.% ci-dessous

\begin{center}
\psset{xunit=5cm,yunit=0.025cm,labelFontSize=\scriptstyle,comma=true,labelsep=0.1pt}
\begin{pspicture}(-0.4,-20)(2.50,260)
\multido{\n=0.0+0.1}{23}{\psline[linewidth=0.3pt,linecolor=lightgray](\n,-20)(\n,260)}
\multido{\n=0+20}{14}{\psline[linewidth=0.3pt,linecolor=lightgray](0,\n)(2.1,\n)}
\psaxes[linewidth=0.95pt,Dx=0.5,Dy=20]{->}(0,0)(0,0)(2.1,260)
\psplot[linewidth=1.25pt,linecolor=blue,plotpoints=5000]{0}{2.1}{2.71828 x 6 mul neg   exp 200 mul 25 add}
\uput[d](1.75,-15){\footnotesize Durée en heure}
\uput[r](0,250){\footnotesize Température en degré Celsius}
\uput[ur](0.3,60){\blue $\mathcal{C}_f$}
\end{pspicture}
\end{center}

\item Avec la précision permise par le graphique, lire $\mathcal{T}_0$. On donnera une valeur approchée de $\mathcal{T}_0$ sous forme d’un nombre entier de minutes.

\item On s’intéresse ici à la diminution, minute après minute, de la température d’une baguette à sa sortie du four.

Ainsi, pour un entier naturel $n$, $\mathcal{D}_n$ désigne la diminution de température en degré Celsius d’une baguette entre la $n$-ième et la $(n+1)$-ième minute après sa sortie du four.

On admet que, pour tout entier naturel $n$ : 

\[\mathcal{D}_n=f\left(\dfrac{n}{60}\right)-f\left(\dfrac{n+1}{60}\right).\]

\smallskip

	\begin{enumerate}
		\item Vérifier que 19 est une valeur approchée de $\mathcal{D}_0$ à 0,1 près, et interpréter ce résultat dans le contexte de l’exercice.
\item Vérifier que l’on a, pour tout entier naturel $n$: 

\[\mathcal{D}_n = 200 \e^{-0,1n}\left(1 - \e^{-0,1}\right).\]

En déduire le sens de variation de la suite $\left(\mathcal{D}_n\right)$, puis la limite de la suite $\left(\mathcal{D}_n\right)$.

Ce résultat était-il prévisible dans le contexte de l’exercice ?

		\end{enumerate}
\end{enumerate}
