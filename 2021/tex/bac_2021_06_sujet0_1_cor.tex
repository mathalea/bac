\textbf{\large\textsc{Exercice 1} \hfill commun à tous les candidats  \hfill 5 points}  

%\medskip
%
%
%
%\emph{Cet exercice est un questionnaire à choix multiples. Pour chacune des questions suivantes, une seule des quatre réponses proposées est exacte. Une réponse exacte rapporte un point. Une réponse fausse, une réponse multiple ou l'absence de réponse à une question ne rapporte ni n’enlève de point. Pour répondre, indiquer sur la copie le numéro de la question et la lettre de la réponse choisie. Aucune justification n’est demandée.}

\medskip

\begin{enumerate}
\item  On considère les suites $(u_n)$ et $(v_n)$ telles que, pour tout entier naturel $n$,

\[u_n=1-\left(\frac{1}{4}\right)^n \hspace{3em} \text{et }\hspace{3em} v_n=1+\left(\frac{1}{4}\right)^n.\]
On considère de plus une suite $(w_n)$ qui, pour tout entier naturel $n$, vérifie $u_n\leqslant w_n\leqslant v_n$.

On peut affirmer que :

\begin{tabularx}{\linewidth}{*{2}{X}}
\textbf{a.~~} Les suites $\left(u_n\right)$ et $\left(v_n\right)$ sont géométriques. &\fbox{\textbf{b.~~}La suite $(w_n)$ converge vers 1. }\\
\textbf{c.~~}La suite $\left(u_n\right)$ est minorée par 1.& \textbf{d.~~} La suite $\left(w_n\right)$ est croissante. 
\end{tabularx}

\begin{tabular}{@{\hspace*{0.05\linewidth}} || p{0.93\linewidth}}
Application directe du théorème dit \og des gendarmes \fg{}.
\end{tabular}

\item On considère la fonction $f$ définie sur $\R$ par : $f(x)= x\e^{x^2}$.

La fonction dérivée de $f$ est la fonction $f'$ définie sur $\R$ par :

\begin{tabularx}{\linewidth}{*{2}{X}}
\textbf{a.~~} $ f'(x) = 2x\e^{x^2}$&\textbf{b.~~} $ f'(x) = (1+2x) \e^{x^2}$\\
\fbox{\textbf{c.~~}$ f'(x) = (1+2x^2) \e^{x^2}$} &\textbf{d.~~} $ f'(x) = (2+x^2) \e^{x^2} $.  
\end{tabularx}

\begin{tabular}{@{\hspace*{0.05\linewidth}} || p{0.93\linewidth}}
$f'(x)= 1\times \e^{x^2} + x \times 2x \e^{x^2} = \left ( 1+2x^2\right )\e^{x^2}$
\end{tabular}


\item Que vaut $\displaystyle \lim_{x\to +\infty} \dfrac{x^2-1}{2x^2-2x+1}$?

\begin{tabularx}{\linewidth}{*{4}{X}}
\textbf{a.~~} $-1 $ &\textbf{b.~~} $0$&\fbox{\textbf{c.~~}$\frac{1}{2}$} & \textbf{d.~~} $+\infty  $.
\end{tabularx}

\begin{tabular}{@{\hspace*{0.05\linewidth}} || p{0.93\linewidth}}
$\displaystyle \lim_{x\to +\infty} \dfrac{x^2-1}{2x^2-2x+1}
= \displaystyle \lim_{x\to +\infty} \dfrac{x^2\left (1-\frac{1}{x^2}\right )}{x^2\left (2-\frac{2}{x}+\frac{1}{x^2}\right )}
= \displaystyle \lim_{x\to +\infty} \dfrac{1-\frac{1}{x^2}}{2-\frac{2}{x}+\frac{1}{x^2}}
=\dfrac{1}{2}$
\end{tabular}

\item On considère une fonction $h$ continue sur l'intervalle $[-1~;~1]$ telle que
\[h(-1)=0 \qquad  h(0)=2\qquad  h(1)=0 .\]

On peut affirmer que :

%\begin{enumerate}
\textbf{~~~a.~~}  La fonction $h$ est croissante sur l'intervalle $[-1~;~0]$.

\textbf{~~~b.~~}  La fonction $h$ est positive sur l'intervalle $[-1~;~1]$.

\fbox{\textbf{~~~c.~~}  Il existe au moins un nombre réel $a$ dans l'intervalle [0~;~1] tel que $h(a) =1$.}

\textbf{~~~d.~~}  l'équation $h(x)=1$ admet exactement deux solutions dans l'intervalle $[-1~;~1]$.
%\end{enumerate}

\begin{tabular}{@{\hspace*{0.05\linewidth}} || p{0.93\linewidth}}
Application du théorème des valeurs intermédiaires sur l'intervalle $[0~;~1]$.
\end{tabular}

\item On suppose que $g$ est une fonction dérivable sur l'intervalle $[-4~;~4]$.
On donne ci-contre la représentation graphique de sa \textbf{fonction dérivée} $g'$.

On peut affirmer que :

\begin{minipage}[]{6cm}
%\begin{enumerate}
{\setlength\parindent{0.3cm}
\textbf{a.~~} $g$ admet un maximum en $-2$.

\textbf{b.~~} $g$ est croissante sur l'intervalle [1~;~2].

\fbox{\textbf{c.~~} $g$ est convexe sur l'intervalle [1~;~2].}

\textbf{d.~~} $g$ admet un minimum en 0.}
%\end{enumerate}
\end{minipage}
\hfill
\begin{minipage}[]{7.5cm}
\psset{xunit=0.8cm,yunit=0.8cm,labelFontSize=\scriptstyle}
\begin{pspicture*}(-4.5,-1.2)(4.5,3.5)
\psgrid[subgriddiv=1, gridcolor=lightgray,gridlabels=0](-5,-2)(5,4)
\psaxes[linewidth=0.5pt]{->}(0,0)(-4.1,-1.2)(4.5,3.5)
\psset{linewidth=1.25pt,showpoints=false,linecolor=red}
\psbezier(-4,0)(-3.5,1.5)(-2.8,3)(-2,3)
\psbezier(-2,3)(-1.2,3)(-0.5,1.5)(0,0)
\psbezier(0,0)(0.25,-0.75)(0.75,-1)(1,-1)
\psbezier(1,-1)(1.25,-1)(1.75,-0.75)(2,0)
\psbezier(2,0)(2.25,0.75)(3.7,4)(4,2)
\psdots(-4,0)(-2,3)(0,0)(1,-1)(2,0)(4,2)
\uput[ul](-3,2.5){\red $\mathcal{C}_{g'}$}
\end{pspicture*}
\end{minipage}

\begin{tabular}{@{\hspace*{0.05\linewidth}} || p{0.93\linewidth}}
La fonction $g'$ est croissante sur l'intervalle $[1~;~2]$, donc la fonction $g$ est convexe sur cet intervalle.\end{tabular}

\end{enumerate}


