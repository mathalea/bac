\textbf{\large Exercice 3 \hfill Commun à tous les candidats \hfill 5 points}

%\medskip
%
%\emph{Cet exercice est un questionnaire à choix multiples. Pour chacune des questions suivantes, une seule des quatre réponses proposées est exacte. \\Une réponse exacte rapporte un point. Une réponse fausse, une réponse multiple ou l'absence de réponse à une question ne rapporte ni n'enlève de point.\\ Pour répondre, indiquer sur la copie le numéro de la question et la lettre de la réponse choisie. \\Aucune justification n'est demandée.}

\begin{center}
\psset{unit=1cm}
\begin{pspicture}(-0.5,-0.5)(9,5)
\psline(0,0)(5,0)(8,1.8)%DCB
\pspolygon[linestyle=dashed](0,0)(8,1.8)(3,1.8)(0,0)%BADB
\psline[linestyle=dashed](5,0)(3,1.8)(4,4)(4,0.95)%CASI
\psline(0,0)(4,4)(5,0)(8,1.8)(4,4)%DSCBS
\uput[dl](0,0){D}\uput[dr](5,0){C}\uput[ur](8,1.8){B}\uput[ul](3,1.8){A}
\uput[u](4,4){S}\uput[d](4,0.95){I}
\uput[ul](2,2){K}\uput[ur](4.5,2){L}\uput[ur](6,2.9){M}
\psdots[dotstyle=+,dotangle=30,dotscale=1.85](2,2)(4.5,2)(6,2.9)
\end{pspicture}
\end{center}

SABCD est une pyramide régulière à base carrée ABCD dont toutes les arêtes ont la même longueur.
Le point I est le centre du carré ABCD. 
On suppose que : IC = IB = IS $= 1$.

Les points K, L et M sont les milieux respectifs des arêtes [SD], [SC] et [SB].

%\medskip

\begin{enumerate}
\item Les droites suivantes ne sont pas coplanaires:

\begin{center}
\begin{tabularx}{\linewidth}{*{4}{X}}
\textbf{a.~} (DK) et (SD) &\textbf{b.~} (AS) et (IC) &\fbox{\textbf{c.~} (AC) et (SB)} &\textbf{d.~} (LM) et (AD)
\end{tabularx}
\end{center}

\begin{tabular}{@{\hspace*{0.05\linewidth}} | p{0.93\linewidth}}
\vspace*{-10pt}
\begin{list}{\textbullet}{On procède par élimination.}
\item Les droites (DK) et (SD) sont sécantes en D donc  coplanaires; on élimine \textbf{a.}
\item Les droites (AS) et (IC) sont sécantes en A donc coplanaires; on élimine \textbf{b.}
\item Les droites (LM) et (AD) sont toutes deux parallèles à (BC) donc parallèles entre elles; elles sont donc coplanaires; on élimine \textbf{d.}
\end{list}

\smallskip

\textbf{Réponse c.}
\end{tabular}
\end{enumerate}

Pour les questions suivantes, on se place dans le repère orthonormé de l'espace $\left(\text{ I}~;~ \vect{\text{IC}},~\vect{\text{IB}},~\vect{\text{IS}}\right)$.

Dans ce repère, on donne les coordonnées des points suivants:

\[\text{I}(0~;~0~;~0) \:;\: \text{A}(-1~;~0~;~0) \:;\: \text{B}(0~;~1~;~0) \:;\: \text{C}(1~;~0~;~0); \text{D}(0~;~-1~;~0) \:;\: \text{S}(0~;~0~;~1).\]

\begin{enumerate}[resume]
\item  Les coordonnées du milieu N de [KL] sont:
\begin{center}
\begin{tabularx}{\linewidth}{*{4}{X}}
\textbf{a.~} $\left(\dfrac{1}{4}~;~\dfrac{1}{4}~;~\dfrac{1}{4}\right)$&
\fbox{\textbf{b.~}$\left(\dfrac{1}{4}~;~- \dfrac{1}{4}~;~\dfrac{1}{2}\right)$}&
\textbf{c.~}$\left(-\dfrac{1}{4}~;~\dfrac{1}{4}~;~\dfrac{1}{2}\right)$&
\textbf{d.~}$\left(-\dfrac{1}{2}~;~\dfrac{1}{2}~;~1\right)$
\end{tabularx}
\end{center}

\begin{tabular}{@{\hspace*{0.05\linewidth}} | p{0.93\linewidth}}
\vspace*{-10pt}
\begin{list}{\textbullet}{}
\item Le milieu K de [SD] a pour coordonnées $\left ( 0~;~-\frac{1}{2}~;~\frac{1}{2}\right )$.
\item Le milieu L de [SC] a pour coordonnées $\left ( \frac{1}{2}~;~0~;~\frac{1}{2}\right )$.
\item Le milieu N de [KL] a donc pour coordonnées $\left ( \frac{1}{4}~;~-\frac{1}{4}~;~\frac{1}{2}\right )$.
\end{list}

\smallskip

\textbf{Réponse b.}
\end{tabular}

\medskip

\item  Les coordonnées du vecteur $\vect{\text{AS}}$ sont:
\begin{center}
\begin{tabularx}{\linewidth}{*{4}{X}}
\textbf{a.~} $\begin{pmatrix}1\\1\\0 \end{pmatrix}$&
\fbox{\textbf{b.~}  $\begin{pmatrix}1\\0\\1 \end{pmatrix}$}&
\textbf{c.~} $\begin{pmatrix}2\\1\\-1 \end{pmatrix}$ &
\textbf{d.~} $\begin{pmatrix} 1\\1\\1\end{pmatrix}$
\end{tabularx}
\end{center}

\begin{tabular}{@{\hspace*{0.05\linewidth}} | p{0.93\linewidth}}
\textbf{Réponse b.}
\end{tabular}

\medskip

\item Une représentation paramétrique de la droite (AS) est:

{\footnotesize \begin{center}
\begin{tabularx}{\linewidth}{*{4}{X}}
\textbf{a.~} $\left\{\begin{array}{l c r}x&=&-1-t\\y&=&t\\z&=&-t
\end{array}\right.$

$(t \in \R)$&\textbf{b.~} $\left\{\begin{array}{l c r}x&=&-1+2t\\y&=&0\\z&=&1 + 2t
\end{array}\right.$

$(t \in \R)$&
\fbox{\parbox{2.6cm}{\textbf{c.~} $\left\{\begin{array}{l c r}x&=&t\\y&=&0\\z&=&1+t
\end{array}\right.$\\
$(t \in \R)$}}
&\textbf{d.~} $\left\{\begin{array}{l c r}x&=&-1-t\\y&=&1+t\\z&=&1-t
\end{array}\right.$

$(t \in \R)$
\end{tabularx}
\end{center}}

\begin{tabular}{@{\hspace*{0.05\linewidth}} | p{0.93\linewidth}}
La droite (AS) a pour vecteur directeur $\vectt{AS}\,(1~;~0~;~1)$; la seule représentation qui convienne est la c.

\smallskip

\textbf{Réponse c.}
\end{tabular}

 \medskip
 
\item Une équation cartésienne du plan (SCB) est:

\begin{center}
\begin{tabularx}{\linewidth}{*{4}{X}}
\textbf{a.~} $y+z-1 =0$ &\fbox{\textbf{b.~}$x+y+z- 1=0$} & \textbf{c.~}$x-y+z=0$&
\textbf{d.~}$x+z-1 =0$
\end{tabularx}
\end{center}

\begin{tabular}{@{\hspace*{0.05\linewidth}} | p{0.93\linewidth}}
\vspace*{-10pt}
\begin{list}{\textbullet}{On procède par élimination.}
\item Le plan d'équation $y+z-1=0$ ne contient pas  C\,$(1~;~0~;~0)$; on élimine \textbf{a.}
\item Le plan d'équation $x-y+z=0$ ne contient pas  S\,$(0~;~0~;~1)$; on élimine \textbf{c.}
\item Le plan d'équation $x+z-1=0$ ne contient pas  B\,$(0~;~1~;~0)$; on élimine \textbf{d.}
\end{list}

\smallskip

\textbf{Réponse b.}
\end{tabular}

\end{enumerate}

\bigskip


