
\textbf{Commun à tous les candidats}

\medskip

\emph{Ceci est un questionnaire à choix multiples (QCM).\\
 Pour chacune des questions, une seule des quatre affirmations est exacte.\\
  Le candidat recopiera sur sa copie le numéro de la question et la réponse correspondante.\\
Aucune justification n'est demandée.\\
Une réponse exacte rapporte un point, une réponse fausse ou une absence de réponse ne rapporte ni n'enlève aucun point.}

\medskip

\begin{enumerate}
\item On considère la fonction définie sur $\R$ par 

\[f(x) = x\e^{-2x}.\]

On note $f''$ la dérivée seconde de la fonction $f$.

Quel que soit le réel $x$, \,$f''(x)$ est égal à :
\begin{center}
\begin{tabularx}{\linewidth}{*{4}{X}}
\textbf{a.~~} $(1 - 2x)\e^{-2x}$& \textbf{b.~~} $4(x - 1)\e^{-2x}$&\textbf{c.~~} $4\e^{-2x}$&\textbf{d.~~}
$ (x+2)\e^{-2x}$
\end{tabularx}
\end{center}
\item Un élève de première générale choisit trois spécialités parmi les douze proposées. Le nombre de combinaisons possibles est:

\begin{center}
\begin{tabularx}{\linewidth}{*{4}{X}}
\textbf{a.~~} \np{1728}&\textbf{b.~~} \np{1320}&\textbf{c.~~} \np{220}&\textbf{d.~~} \np{33}
\end{tabularx}
\end{center}
\item On donne ci-dessous la représentation graphique de $f'$ fonction dérivée d'une fonction $f$ définie sur [0~;~7].

\begin{center}
\psset{unit=0.7cm}
\begin{pspicture*}(-1,-4.5)(7.5,1.1)
\psgrid[gridlabels=0pt,gridwidth=0.3pt,subgriddiv=5,subgridwidth=0.1pt](-1,-4.5)(7.5,1)
\psaxes[linewidth=1.25pt,labelFontSize=\scriptstyle](0,0)(-0.95,-4.5)(7.5,1)
\psplot[plotpoints=2000,linewidth=1.25pt,linecolor=blue]{0}{7}{2 x sub x 5 sub mul 0.4 mul 2.71828 x 0.25 mul exp div}
\end{pspicture*}
\end{center}

Le tableau de variation de $f$  sur l'intervalle [0~;~7] est :
\begin{center}
\begin{tabularx}{\linewidth}{*{2}{X}}
\textbf{a.~~}&\textbf{b.~~}\\
\begin{pspicture}(5.5,1.5)
\psframe(5.5,1.5)
\psline(0,1)(5.5,1)\psline(1,0)(1,1.5)
\uput[u](0.5,0.9){$x$}\uput[u](1.15,0.9){0}\uput[u](3.25,0.9){3,25}\uput[u](5.3,0.9){7}
\rput(0.5,0.5){$f(x)$}
\psline{->}(1.4,0.2)(3,0.8)\psline{->}(3.5,0.8)(5.2,0.2)
\end{pspicture}&
\begin{pspicture}(5.5,1.5)
\psframe(5.5,1.5)
\psline(0,1)(5.5,1)\psline(1,0)(1,1.5)
\uput[u](0.5,0.9){$x$}\uput[u](1.15,0.9){0}\uput[u](2.5,0.9){2}\uput[u](4,0.9){5}\uput[u](5.3,0.9){7}
\rput(0.5,0.5){$f(x)$}
\psline{->}(1.2,0.8)(2.3,0.2)\psline{->}(2.7,0.2)(3.8,0.8)\psline{->}(4.2,0.8)(5.3,0.2)
\end{pspicture}\\
\textbf{c.~~}&\textbf{d.~~}\\
\begin{pspicture}(5.5,1.5)
\psframe(5.5,1.5)
\psline(0,1)(5.5,1)\psline(1,0)(1,1.5)
\uput[u](0.5,0.9){$x$}\uput[u](1.15,0.9){0}\uput[u](2.5,0.9){2}\uput[u](4,0.9){5}\uput[u](5.3,0.9){7}
\rput(0.5,0.5){$f(x)$}
\psline{->}(1.2,0.2)(2.3,0.8)\psline{->}(2.7,0.8)(3.8,0.2)\psline{->}(4.2,0.2)(5.3,0.8)
\end{pspicture}&
\begin{pspicture}(5.5,1.5)
\psframe(5.5,1.5)
\psline(0,1)(5.5,1)\psline(1,0)(1,1.5)
\uput[u](0.5,0.9){$x$}\uput[u](1.15,0.9){0}\uput[u](3.25,0.9){2}\uput[u](5.3,0.9){7}
\rput(0.5,0.5){$f(x)$}
\psline{->}(1.4,0.2)(3,0.8)\psline{->}(3.5,0.8)(5.2,0.2)
\end{pspicture}\\
\end{tabularx}
\end{center}

\item Une entreprise fabrique des cartes à puces. Chaque puce peut présenter deux défauts notés A et B.

Une étude statistique montre que 2,8\,\% des puces ont le défaut A, 2,2\,\% des puces ont le défaut B et, heureusement, 95,4\,\% des puces n'ont aucun des deux défauts.

La probabilité qu'une puce prélevée au hasard ait les deux défauts est:

\begin{center}
\begin{tabularx}{\linewidth}{*{4}{X}}
\textbf{a.~~} 0,05&\textbf{b.~~} 0,004&\textbf{c.~~} 0,046&\textbf{d.~~} On ne peut pas le savoir
\end{tabularx}
\end{center}

\item On se donne une fonction $f$, supposée dérivable sur $\R$, et on note $f’$ sa fonction dérivée.

On donne ci-dessous le tableau de variation de $f$ :

\begin{center}
\psset{xunit=1.25cm,yunit=1.25cm}
\begin{pspicture}(5.5,1.5)
\psframe(5.5,1.5)
\psline(0,1)(5.5,1)\psline(1,0)(1,1.5)
\uput[u](0.5,0.9){$x$}\uput[u](1.5,0.9){$- \infty$}\uput[u](3.25,0.9){$-1$}\uput[u](5.1,0.9){$+\infty$}
\rput(0.5,0.5){$f(x)$}
\psline{->}(1.9,0.2)(3,0.8)\psline{->}(3.5,0.8)(4.6,0.2)
\uput[u](1.5,0){$- \infty$}\uput[u](5,0){$- \infty$}
\uput[d](3.25,1){$0$}
\end{pspicture}
\end{center}

D'après ce tableau de variation :

\textbf{a.~~} $f’$ est positive sur $\R$.

\textbf{b.~~} $f’$ est positive sur $] - \infty~;~ - 1]$ 

\textbf{c.~~} $f’$ est négative sur $\R$

\textbf{d.~~} $f’$ est positive sur $[- 1~;~ +\infty[$
\end{enumerate}


\bigskip

