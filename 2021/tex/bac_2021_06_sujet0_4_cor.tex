\textbf{\large\textsc{Exercice A} \hfill exercice au choix \hfill 5 points}  

\medskip
\framebox{
\begin{minipage}[]{5.25cm}
\textbf{Principaux domaines abordés}\\
Logarithme\\
Dérivation, convexité, limites
\end{minipage}}

\medskip

Sur le graphique ci-dessous, on a représenté dans un repère orthonormé :
\begin{list}{\textbullet}{}
\item la courbe représentative $\mathcal{C}_f$ d'une fonction $f$ définie et dérivable sur $]0~;~+\infty[$ ;
\item la tangente $\mathcal{T}_A$ à la courbe $\mathcal{C}_f$ au point A de coordonnées $\left(\dfrac{1}{\e}~;~\e\right)$ ;
\item la tangente $\mathcal{T}_B$ à la courbe $\mathcal{C}_f$ au point B de coordonnées (1~;~2).
\end{list}

La droite $\mathcal{T}_A$ est parallèle à l'axe des abscisses. La droite $\mathcal{T}_B$ coupe l'axe des abscisses au point de coordonnées (3~;~0) et l'axe des ordonnées au point de coordonnées (0~;~3).

\begin{center}

\psset{xunit=1.8cm,yunit=1.8cm,labelFontSize=\scriptstyle,comma=true}
\begin{pspicture}(-0.4,-0.7)(8,3.7)
\multido{\n=0+0.5}{16}{\psline[linewidth=0.35pt,linecolor=lightgray](\n,-0.5)(\n,3.4)}
%\multido{\n=0+1}{51}{\psline[linewidth=0.3pt](\n,-100)(\n,100)}
\multido{\n=-0.5+0.5}{8}{\psline[linewidth=0.35pt,linecolor=lightgray](-0.4,\n)(7.50,\n) }
\psaxes[linewidth=0.95pt,Dx=0.5,Dy=0.5]{->}(0,0)(-0.3,-0.7)(7.6,3.6)
\psplot[linewidth=1.25pt,linecolor=blue,plotpoints=5000]{0.125}{7.50}{x ln 2 add x div}
\psplot[linewidth=0.85pt,linecolor=cyan,plotpoints=5000]{-0.1}{7.50}{2.71828}
\psplot[linewidth=0.85pt,linecolor=cyan,plotpoints=5000]{-0.1}{3.50}{x neg 3 add}
\psdots[dotstyle=Bullet,dotscale =1.1](0.367879,2.71828)(1,2)
\uput[ur](0.4,2.71828){A}\uput[ur](1,2){B}
\uput[d](6.25,2.7){\cyan $\mathcal{T}_A$}
\uput[dl](2.5,0.5){\cyan $\mathcal{T}_B$}
\uput[r](5,0.5) {\blue $\mathcal{C}_f$}
\end{pspicture}
\end{center}

On note $f'$ la fonction dérivée de $f$.

\textbf{\textsc{Partie} I}

\begin{enumerate}
\item  %Déterminer graphiquement les valeurs de $f'\left(\dfrac{1}{\e}\right)$ et de $f'(1)$.
\begin{list}{\textbullet}{}
\item  La droite $\mathcal{T}_A$ est tangente à la courbe $\mathcal{C}_f$ au point A de coordonnées $\left(\dfrac{1}{\e}~;~\e\right)$; elle a donc comme coefficient directeur $f'\left (\dfrac{1}{\e}\right )$.

La droite $\mathcal{T}_A$ est parallèle à l'axe des abscisses donc son coefficient directeur est nul.

On peut donc déduire que $f'\left (\dfrac{1}{\e}\right )=0$.
\item La droite $\mathcal{T}_B$ est tangente à la courbe $\mathcal{C}_f$ au point B de coordonnées (1~;~2), donc elle a pour coefficient directeur $f'(1)$.

La droite $\mathcal{T}_B$ coupe l'axe des abscisses au point de coordonnées (3~;~0) et l'axe des ordonnées au point de coordonnées (0~;~3), donc on peut en déduire que son coefficient directeur est $\dfrac{3-0}{0-3}=-1$.

On a donc $f'(1)=-1$.
\end{list}

\item La droite $\mathcal{T}_B$ a pour coefficient directeur $-1$ et 3 pour ordonnée à l'origine, donc elle a pour équation: $y=-x+3$.
\end{enumerate}

\medskip

\textbf{\textsc{Partie} II}

On suppose maintenant que la fonction $f$ est définie sur $]0~;~+\infty[$ par : 
$f(x)=\dfrac{2+\ln(x)}{x}$.

\begin{enumerate}
\item %Par le calcul, montrer que la courbe $\mathcal{C}_f$ passe par les points A et B et qu'elle coupe l'axe des abscisses en un point unique que l'on précisera.
\begin{list}{\textbullet}{}
\item $f\left (\dfrac{1}{\e}\right ) = \dfrac{2+\ln\left (\frac{1}{\e}\right )}{\frac{1}{\e}}
= \e\left ( 2 - \ln(\e)\right ) = = \e\left ( 2 - 1\right )=\e$ donc $\text A \in \mathcal{C}_f$.
\item $f(1)=\dfrac{2+\ln(1)}{1}=2$ donc $\text B \in \mathcal{C}_f$.
\item La courbe $\mathcal{C}_f$ coupe l'axe des abscisses en un point dont l'abscisse est solution de l'équation $f(x)=0$. On résout dans $]0~;~+\infty[$ cette équation.

$f(x)=0 \iff \dfrac{2+\ln(x)}{x} = 0
\iff 2+\ln(x)=0 \iff \ln(x)=-2 \iff x = \e^{-2}$

Donc la courbe $\mathcal{C}_f$ coupe l'axe des abscisses en un point unique de coordonnées $\left ( \e^{-2}~;~0\right )$.
\end{list}

\item %Déterminer la limite de $f(x)$ quand $x$ tend vers 0 par valeurs supérieures, et la limite de $f(x)$ quand $x$ tend vers $+\infty$.
Calculs des limites.

\begin{list}{}{}
\item $\left .
\begin{array}{r}
\ds\lim_{x \to 0 \atop x>0} \left (2+\ln(x)\right )=-\infty\\
\ds\lim_{x \to 0 \atop x>0} \dfrac{1}{x}=+\infty
\end{array}
\right \rbrace
\implies
\ds\lim_{x \to 0 \atop x>0} \left (2+\ln(x)\right )\times \dfrac{1}{x} = -\infty
\text{ donc } \ds\lim_{x \to 0 \atop x>0} f(x)=-\infty$
\item $\left .
\begin{array}{r}
\ds\lim_{x \to +\infty} \dfrac{2}{x}=0\rule[-10pt]{0pt}{0pt}\\
\ds\lim_{x \to +\infty} \dfrac{\ln(x)}{x}=0
\end{array}
\right \rbrace
\implies
\ds\lim_{x \to +\infty} \dfrac{2}{x} + \dfrac{\ln(x)}{x}=0
\text{ donc } \ds\lim_{x \to +\infty} f(x)=0$
\end{list}

\item %Montrer que, pour tout $x\in]0~;~\infty[$, $f'(x)=\dfrac{-1-\ln(x)}{x^2} .$
Pour $x\in]0~;~\infty[$, $f'(x)=\dfrac{\frac{1}{x}\times x - (2+\ln(x)) \times 1}{x^2} = \dfrac{1-2-\ln(x)}{x^2} = \dfrac{-1-\ln(x)}{x^2}$.


\item $f'(x)$ est du signe de $-1-\ln(x)$;
$-1-\ln(x) >0 \iff -1 > \ln(x) \iff x < \e^{-1}$

On dresse le tableau de variations de $f$ sur $]0~;~+\infty[$:

\begin{center}
{\renewcommand{\arraystretch}{1.3}
\psset{nodesep=3pt,arrowsize=2pt 3}%  paramètres
\def\esp{\hspace*{2.5cm}}% pour modifier la largeur du tableau
\def\hauteur{0pt}% mettre au moins 20pt pour augmenter la hauteur
$\begin{array}{|c|l *4{c}|}
\hline
x & 0  & \esp & \frac{1}{\e} & \esp & +\infty \\ 
\hline
f'(x) &\vline\;\vline  &   \pmb{+} & \vline\hspace{-2.7pt}0 & \pmb{-} & \\ 
\hline
 & \vline\;\vline &  &   \Rnode{max}{\e}  &  &   \\  
f(x) &\vline\;\vline &     &  &  &  \rule{0pt}{\hauteur} \\ 
 & \vline\;\vline \Rnode{min1}{~-\infty} &   &  &  &   \Rnode{min2}{0} \rule{0pt}{\hauteur}    
 \ncline{->}{min1}{max} 
 \ncline{->}{max}{min2} 
 \\ 
\hline
\end{array} $
}
\end{center}	

\item %On note $f''$ la fonction dérivée seconde de $f$.
On admet que, pour tout $x\in]0~;~+\infty[$, $f''(x)=\dfrac{1+2\ln(x)}{x^3} $.

%Déterminer le plus grand intervalle sur lequel $f$ est convexe.

La fonction $f$ est convexe sur les intervalles sur lesquels $f''$ est positive.

Sur $]0~;~+\infty$, $x^3>0$ donc

$f''(x)\geqslant 0
\iff \dfrac{1+2\ln(x)}{x^3} \geqslant 0
\iff 1+2\ln(x) \geqslant 0
\iff \ln(x) \geqslant -\dfrac{1}{2}
\iff x\geqslant \e^{-\frac{1}{2}}$

Donc le plus grand intervalle sur lequel la fonction $f$ est convexe est $\left [\e^{-\frac{1}{2}}~;~+\infty \right [$.
\end{enumerate}


