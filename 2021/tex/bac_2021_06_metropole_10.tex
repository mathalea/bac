
\smallskip

\begin{tabular}{|l|}\hline
Principaux domaines abordés :\\
Fonction exponentielle; dérivation; convexité\\ \hline
\end{tabular}

\begin{center}\textbf{Partie 1}\end{center}

On donne ci-dessous, dans le plan rapporté à un repère orthonormé, la courbe représentant la
fonction dérivée $f'$ d'une fonction $f$ dérivable sur $\R$.

À l'aide de cette courbe, conjecturer, en justifiant les réponses : 

\medskip

\begin{enumerate}
\item Le sens de variation de la fonction $f$ sur $\R$.
\item La convexité de la fonction $f$ sur $\R$.
\end{enumerate}

\begin{center}
\psset{unit=1cm,arrowsize=2pt 3}
\begin{pspicture*}(-3.25,-2)(5.25,4.5)
\psgrid[gridlabels=0pt,subgriddiv=1,gridwidth=0.15pt](-2,-1)(4,4.5)
\psaxes[linewidth=1.25pt](0,0)(-2,-1.25)(4,4.5)
\psaxes[linewidth=1.25pt]{->}(0,0)(1,1)
\psplot[plotpoints=2000,linewidth=1.25pt,linecolor=red]{-2}{4}{1 x add 2.71828 x exp div  neg}
\rput(1,-1.5){Courbe représentant la \textbf{dérivée} $f'$ de la fonction $f$.}
\end{pspicture*}
\end{center}

\smallskip

\begin{center}\textbf{Partie 2}\end{center}

\smallskip

On admet que la fonction $f$ mentionnée dans la Partie 1 est définie sur $\R$ par :

\[f(x) = (x + 2)\e^{-x}.\]

On note $\mathcal{C}$ la courbe représentative de $f$ dans un repère orthonormé \Oij.

On admet que la fonction $f$ est deux fois dérivable sur $\R$, et on note $f'$ et $f''$ les fonctions dérivées première et seconde de $f$ respectivement.

\medskip

\begin{enumerate}
\item Montrer que, pour tout nombre réel $x$,

\[f(x) = \dfrac{x}{\e^{x}}+ 2\e^{-x}.\]

En déduire la limite de $f$ en $+ \infty$.

Justifier que la courbe $\mathcal{C}$ admet une asymptote que l'on précisera.

On admet que $\displaystyle\lim_{x \to - \infty} f(x) = - \infty$.
\item 
	\begin{enumerate}
		\item Montrer que, pour tout nombre réel $x$,\, $f'(x) = (- x - 1)\e^{-x}$.
		\item Étudier les variations sur $\R$ de la fonction $f$ et dresser son tableau de variations.
		\item Montrer que l'équation $f(x) = 2$ admet une unique solution $\alpha$ sur l'intervalle $[-2~;~-1]$ dont on donnera une valeur approchée à $10^{-1}$ près.
	\end{enumerate}
\item Déterminer, pour tout nombre réel $x$, l'expression de $f''(x)$ et étudier la convexité de la fonction~$f$. 

Que représente pour la courbe $\mathcal{C}$ son point A d'abscisse $0$ ?
\end{enumerate}
