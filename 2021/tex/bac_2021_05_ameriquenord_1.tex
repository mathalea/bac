
\textbf{Commun à tous les candidats}

\medskip


Les probabilités demandées dans cet exercice seront arrondies à $10^{-3}$.

\medskip

Un laboratoire pharmaceutique vient d'élaborer un nouveau test anti-dopage. 

\bigskip

\textbf{Partie A}

\medskip

Une étude sur ce nouveau test donne les résultats suivants:

\setlength\parindent{9mm}
\begin{itemize}
\item[$\bullet~~$] si un athlète est dopé, la probabilité que le résultat du test soit positif est $0,98$ (sensibilité du test) ;
\item[$\bullet~~$]si un athlète n'est pas dopé, la probabilité que le résultat du test soit négatif est $0,995$ (spécificité du test).
\end{itemize}
\setlength\parindent{0mm}

\smallskip

On fait subir le test à un athlète sélectionné au hasard au sein des participants à une compétition d'athlétisme. 

On note $D$ l'évènement \og l'athlète est dopé \fg{} et $T$ l'évènement \og le test est positif \fg. 

On admet que la probabilité de l'évènement $D$ est égale à 0,08.

\medskip

\begin{enumerate}
\item Traduire la situation sous la forme d'un arbre pondéré.
\item Démontrer que $P(T) = 0,083$.
\item 
	\begin{enumerate}
		\item Sachant qu'un athlète présente un test positif, quelle est la probabilité qu'il soit dopé ?
		\item Le laboratoire décide de commercialiser le test si la probabilité de l'évènement \og un athlète présentant un test positif est dopé \fg{} est supérieure ou égale à $0,95$.

Le test proposé par le laboratoire sera-t-il commercialisé ? Justifier.
	\end{enumerate}
\end{enumerate}

\bigskip

\textbf{Partie B}

\medskip

Dans une compétition sportive, on admet que la probabilité qu'un athlète contrôlé présente un test positif est $0,103$.

\medskip

\begin{enumerate}
\item Dans cette question \textbf{1.} on suppose que les organisateurs décident de contrôler $5$ ~athlètes au hasard parmi les athlètes de cette compétition. 

On note $X$ la variable aléatoire égale au nombre d'athlètes présentant un test positif parmi les $5$ athlètes contrôlés.
	\begin{enumerate}
		\item Donner la loi suivie par la variable aléatoire $X$. Préciser ses paramètres.
		\item Calculer l'espérance $E(X)$ et interpréter le résultat dans le contexte de l'exercice.
		\item Quelle est la probabilité qu'au moins un des 5 athlètes contrôlés présente un test positif ?
	\end{enumerate}
\item Combien d'athlètes faut-il contrôler au minimum pour que la probabilité de l'évènement \og au moins un athlète contrôlé présente un test positif\fg{} soit supérieure ou égale à $0,75$ ? Justifier.
\end{enumerate}
\vspace{0,5cm}
\bigskip

