
\textbf{Commun à tous les candidats}

\medskip

\emph{Cet exercice est un questionnaire à choix multiples. Pour chacune des questions suivantes, une seule des quatre réponses proposées est exacte. \\
Une réponse exacte rapporte un point. Une réponse fausse, une réponse multiple ou l'absence de réponse à une question ne rapporte ni n'enlève de point.\\
Pour répondre, indiquer sur la copie le numéro de la question et la lettre de la réponse choisie.\\
Aucune justification n'est demandée.}

\medskip

L'espace  est rapporté à un repère orthonormé \Oijk. 

On considère :

\setlength\parindent{9mm}
\begin{itemize}
\item[$\bullet~~$]La droite $\mathcal{D}$ passant par les points A$(1~;~1~;~ -2)$ et B$(-1~;~3~;~2)$. 
\item[$\bullet~~$]La droite $\mathcal{D}'$ de représentation paramétrique :
$\left\{\begin{array}{l c l}
x &=& -4 +3t\\
y&=&\phantom{-}6 - 3t\\
z&=&\phantom{-}8 - 6t
\end{array}\right.$\quad avec $t \in \R$.
\item[$\bullet~~$]Le plan $\mathcal{P}$ d'équation cartésienne $x + my- 2z + 8 = 0$ où $m$ est un nombre réel.
\end{itemize}
\setlength\parindent{0mm}

\medskip

\textbf{Question 1} : Parmi les points suivants, lequel appartient à la droite $\mathcal{D}'$ ?

\begin{center}
\begin{tabularx}{\linewidth}{*{4}{X}}
\textbf{a.~~} M$_1(-1~;~3~;~-2)$&\textbf{b.~~} M$_2(11~;~-9~;~-22)$&\textbf{c.~~} M$_3(-7~;~9~;~ 2)$&\textbf{d.~~}M$_4(-2~;~3~;~4)$
\end{tabularx}
\end{center}
\textbf{Question 2} : Un vecteur directeur de la droite $\mathcal{D}'$est :
\begin{center}
\begin{tabularx}{\linewidth}{*{4}{X}}
\textbf{a.~~}$\vect{u_1}\begin{pmatrix}-4\\6\\8\end{pmatrix}$&\textbf{b.~~}$\vect{u_2}\begin{pmatrix}3\\3\\6\end{pmatrix}$&\textbf{c.~~}$\vect{u_3}\begin{pmatrix}3\\-3\\-6\end{pmatrix}$&\textbf{d.~~}$\vect{u_4}\begin{pmatrix}-1\\3\\2\end{pmatrix}$
\end{tabularx}
\end{center}
\textbf{Question 3} : Les droites $\mathcal{D}$ et $\mathcal{D}'$ sont :
\begin{center}
\begin{tabularx}{\linewidth}{*{4}{X}}
\textbf{a.~~}sécantes &\textbf{b.~~} strictement parallèles&\textbf{c.~~}non coplanaires&\textbf{d.~~}
confondues
\end{tabularx}
\end{center}
\textbf{Question 4} : La valeur du réel $m$ pour laquelle la droite $\mathcal{D}$ est parallèle au plan $\mathcal{P}$ est :
\begin{center}
\begin{tabularx}{\linewidth}{*{4}{X}}
\textbf{a.~~}$m =-1$ &\textbf{a.~~} $m = 1$&\textbf{c.~~}$m = 5$&\textbf{d.~~}$m =-2$
\end{tabularx}
\end{center}

\bigskip

